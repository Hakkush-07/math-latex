\documentclass[12pt]{scrartcl}
\usepackage{hakan}

% \usepackage[inline]{asymptote}
\usepackage{tikz}
\usetikzlibrary{arrows.meta}

\newcommand{\hgline}[2]{
\pgfmathsetmacro{\thetaone}{#1}
\pgfmathsetmacro{\thetatwo}{#2}
\pgfmathsetmacro{\theta}{(\thetaone+\thetatwo)/2}
\pgfmathsetmacro{\phi}{abs(\thetaone-\thetatwo)/2}
\pgfmathsetmacro{\close}{less(abs(\phi-90),0.0001)}
\ifdim \close pt = 1pt
    \draw[blue] (\theta+180:1) -- (\theta:1);
\else
    \pgfmathsetmacro{\R}{tan(\phi)}
    \pgfmathsetmacro{\distance}{sqrt(1+\R^2)}
    \draw[blue] (\theta:\distance) circle (\R);
\fi
}

\newcommand\anglex{10}

\renewcommand{\H}{\mathbb{H}}
\DeclareMathOperator{\PSL}{PSL}

\title{Geometrically Finite Fuchsian Groups and Poincare Theorem}

\begin{document}

\maketitle

\begin{abstract}
    
\end{abstract}

\newpage

\tableofcontents

\newpage

\section{Introduction}

In this write-up, we will look at geometrically finite Fuchsian groups and the Poincare theorem.

\begin{maintheorem*}[Poincare]
    There exists a Fuchsian group with signature $(g;m_1,m_2,\cdots,m_r;s)$ if $g\geq 0$, $r\geq 0$, $m_i\geq 2$, $s\geq 0$ are integers and $$(2g-2)+\sum_{i=1}^r\left(1-\frac{1}{m_i}\right)+s>0$$
\end{maintheorem*}

First, some notions such as cycles and periods of the fundamental domains of Fuchsian groups are defined. Then, we look at what being geometrically finite means for a Fuchsian group. Finally, we define the signatures, which compactly represent the Fuchsian groups, and we see that almost all signatures are possible to be attained by a Fuchsian group.

\section{The Structure of Dirichlet Domains}

\begin{definition*}[sides, vertices, faces]
    Dirichlet domains are bounded by geodesics and possibly segments of real axis. These bounding geodesic segments are called $\vocab{sides}$, and the intersection of them, together with elliptic points of order 2 on the segments, are called $\vocab{vertices}$. Tessellation of $\mathbb{H}$ by a Dirichlet domain $F$, $\{T(F):T\in\Gamma\}$, consists of the images of $F$ under $\Gamma$, which are called \vocab{faces}.
\end{definition*}

\begin{definition*}[congruence, cycles, periods]
    Two points $u,v\in\mathbb{H}$ are called \vocab{congruent} if they belong to the same orbit of $\Gamma$. The congruence is an equivalence relation on the vertices of a Dirichlet domain and the equivalence classes are called \vocab{cycles}. If $u$ is fixed by an elliptic element $S$, then $v=Tu$ is fixed by $TST^{-1}$. Thus, if one vertex of a cycle is fixed by an elliptic element, then all vertices of that cycle are fixed by conjugate elliptic elements. Such a cycle is called \vocab{elliptic cycle}, vertices are called \vocab{elliptic vertices}, and the number of elliptic cycles is equal to the non-congruent elliptic points in $F$. If a point $w\in\H$ has a non-trivial stabilizer, then this stabilizer is a maximal finite cyclic subgroup of $\Gamma$. The order of non-conjugate maximal finite cyclic subgroups of $\Gamma$ are called \vocab{periods}.
\end{definition*}

\begin{remark*}
    There is a one-to-one correspondence between the elliptic cycles of $F$ and the conjugacy classes of non-trivial maximal finite cyclic subgroups of $\Gamma$.
\end{remark*}

\begin{example*}
    Take $\Gamma=\PSL_2(\mathbb{Z})$, the modular group, and take $F$ to be the fundamental domain in the figure below.
    \begin{center}
    \begin{tikzpicture}[scale=1.6]
    \draw[very thick,fill=gray!30] (0.5, 3.0) -- (0.5, 0.8660254037844386) arc (59.99999999999999:120.00000000000001:0.9999999999999999) -- (-0.5, 3.0);
    
    \draw[-latex] (-0.7-0.5,0) -- (0.7+0.5,0)node[below]{Re};
    \draw[-latex] (0,0) -- (0,3.0+0.5)node[right]{Im};
    \path(-1,0) --node[below, pos=0]{$-1$}node[below right, pos=.5]{0}node[below, pos=1]{1} (1,0) (0,1)node[below right]{$i$};
    \end{tikzpicture}
    \end{center}

    Vertices of $F$ are $w,w-1,i,\infty$ where $w=\frac{1}{2}+\frac{\sqrt{3}}{2}i$. Any point $u$ on the left side is congruent to $u+1$ on the right side via $z\mapsto z+1$. $w,w-1,i$ are fixed by the cyclic finite groups generated by $z\mapsto\frac{z-1}{z}$, $z\mapsto\frac{-z-1}{z}$, and $z\mapsto\frac{-1}{z}$ respectively. $w$ and $w-1$ are congruent vertices, so $\{w,w-1\}$ and $\{i\}$ are the elliptic cycles. Non-conjugate maximal finite cyclic subgroups of $\Gamma$ are $\{1,S\}$ and $\{1,U,U^2\}$ where $S:z\mapsto \frac{-1}{z}$, $T:z\mapsto z+1$, and $U=ST:z\mapsto \frac{-1}{z+1}$. A parabolic element can be considered as an infinite order elliptic element, hence the stabilizer of an element $w\in\overline{\mathbb{R}}$ is a maximal cyclic parabolic subgroup. If we allow infinite periods, then since $\{1,T,T^2,\cdots,\}$ is a maximal cyclic parabolic subgroup of $\Gamma$, the modular group has periods $2,3,\infty$.
\end{example*}

\begin{theorem}\label{thm:2pim}
    Let $F$ be a Dirichlet domain for $\Gamma$. Let $\theta_1,\theta_2,\cdots,\theta_t$ be the internal angles at a cycle of $F$ (there are finitely many points in a congruent cycle because $F$ is locally finite). Let $m$ be the order of the stabilizer of one of these vertices (stabilizers of two points in a congruent cycle are conjugate subgroups of $\Gamma$, so they have the same order). Then $$\theta_1+\theta_2+\cdots+\theta_t=\frac{2\pi}{m}$$
\end{theorem}

\begin{proof}
    Let $v_1,v_2,\cdots,v_t$ be the vertices of the congruent cycle, $\theta_1,\theta_2,\cdots,\theta_t$ be the angles, and $$H=\{\text{Id},S,S^2,\cdots,S^{m-1}\}$$ be the stabilizer of $v_1$. Each $S^r(F)$ has a vertex at $S^r(v_1)=v_1$ with angle $\theta_1$. Now we will look at other vertices being sent to $v_1$. Suppose $T_k(v_k)=v_1$, then each $S^rT_k(F)$ has a vertex at $S^rT_k(v_k)=v_1$ with angle $\theta_k$. We have $mt$ angles surrounding $v_1$, which add up to $2\pi$.
\end{proof}

\begin{example*}
    In the modular group, $\{w,w-1\}$ is a congruent cycle, and the sum of angles at these vertices is $\frac{\pi}{3}+\frac{\pi}{3}=\frac{2\pi}{3}$ as $m=3$. $\{i\}$ is a congruent cycle, and the angle is $\pi=\frac{2\pi}{2}$ as $m=2$.
\end{example*}

Sides can also be congruent. For a side $s$ and $T\in\Gamma$, if $T(s)$ is a side too, then $s$ and $T(s)$ are called \vocab{congruent sides}. Sides of $F$ fall into congruent pairs.

\begin{theorem}\label{thm:genset}
    The subset of $\Gamma$ consisting of elements pairing the sides of $F$ is a generator set for $\Gamma$.
\end{theorem}

\begin{proof}
    Let $\Gamma'$ be the group generated by the elements pairing the sides of $F$. Take $S\in\Gamma'$. For $U,V\in\Gamma$, such that $S(F)$ and $U(F)$ share a side, and $S(F)$ and $V(F)$ share a vertex, $S^{-1}U$ shares a side with $F$ and $S^{-1}V$ shares a vertex with $F$. $S^{-1}U\in\Gamma'\implies U\in\Gamma'$ and we can go to $S^{-1}V$ from $F$ by following finitely many faces sharing a side, which means $S^{-1}V\in\Gamma'\implies V\in\Gamma'$. Now we have $U,V\in\Gamma'$, so $X=\bigcup_{T\in\Gamma'}T(F)$ and $Y=\bigcup_{T\in\Gamma-\Gamma'}T(F)$ are disjoint and $X\cup Y=\H$. Any union of faces of the tessellation is closed since $F$ is locally finite, and $\H$ is connected. Therefore, $X=\H$ and $Y=\emptyset$.
\end{proof}

\begin{example*}
    In the modular group, the sides $\{\frac{1}{2}+ci:c^2\geq\frac{3}{4}\}$ and $\{\frac{-1}{2}+ci:c^2\geq\frac{3}{4}\}$ are congruent via $z\mapsto z+1$. The sides $\{(\cos(\alpha),\sin(\alpha):\frac{\pi}{3}\leq\alpha\leq\frac{\pi}{2})\}$ and $\{(\cos(\alpha),\sin(\alpha):\frac{\pi}{2}\leq\alpha\leq\frac{2\pi}{3})\}$ are congruent via $z\mapsto \frac{-1}{z}$. So, the modular group is generated by $\{z\mapsto z+1,z\mapsto \frac{-1}{z}\}$.
\end{example*}

\begin{remark*}
    Let $\Gamma$ be a Fuchsian group with $\mu(\H/\Gamma)<\infty$ and $F$ be a fundamental domain for it. The restriction of $\pi:\H\rightarrow\H/\Gamma$, $z\mapsto$ the orbit of $z$, to $F$, identifies the congruent points of $F$ that necessarily belong to $\partial F$, and makes $\H/\Gamma$ an oriented surface with possibly some marked points, corresponding to elliptic cycles, and cusps, corresponding to non-congruent vertices at infinity, and this is an \vocab{orbifold}.
\end{remark*}

\section{Geometry of Fuchsian Groups}

\begin{definition*}[geometrically finite]
    A Fuchsian group is called \vocab{geometrically finite} if there exists a convex fundamental domain for $\Gamma$ with finitely many sides.
\end{definition*}

\begin{theorem*}[Siegel]
    A Fuchsian group $\Gamma$ with a finite area is geometrically finite.
\end{theorem*}

\begin{proof}
    Look at Theorem 4.1.1 in \cite{ref:katok} (Katok).
\end{proof}

\begin{definition*}[cocompact]
    A Fuchsian group $\Gamma$ is called \vocab{cocompact} if equivalently one of the following is true:
    \begin{itemize}
        \item $\H/\Gamma$ is compact
        \item any Dirichlet domain $F$ for $\Gamma$ is compact
        \item $\mu(\H/\Gamma)$ is finite and $\Gamma$ does not contain parabolic elements
    \end{itemize}
\end{definition*}

\begin{remark*}
    There is a one-to-one correspondence between non-congruent vertices at infinity of a Dirichlet fundamental domain for a non-cocompact Fuchsian group $\Gamma$ with finite $\mu(\H/\Gamma)$ and conjugacy classes of maximal parabolic subgroups of $\Gamma$.
\end{remark*}

\begin{remark*}
    Compact fundamental domains have finitely many sides. Non-compact fundamental domains with finite area have at least one vertex at infinity.
\end{remark*}

\section{The Signature of a Fuchsian Group}

\begin{definition*}[signature]
    For one case, assume that $\Gamma$ is cocompact. It has finitely many sides, vertices, elliptic cycles, and periods $m_1,m_2,\cdots,m_r$. Also, $\H/\Gamma$ is an orbifold, a compact oriented surface of genus $g$. In this case, we say $\Gamma$ has a \vocab{signature} $(g;m_1,m_2,\cdots,m_r)$.

    For the other case, assume that $\Gamma$ is non-cocompact. Assume that $\Gamma$ has $r$ conjugacy classes of maximal elliptic cyclic subgroups of order $m_1,m_2,\cdots,m_r$ and has $s$ conjugacy classes of maximal parabolic cyclic subgroups. Also, $\H/\Gamma$ is an orbifold with genus $g$. In this case, we say $\Gamma$ has a \vocab{signature} $(g;m_1,m_2,\cdots,m_r;s)$. $s=0$ can be considered as the first case.
\end{definition*}

\begin{theorem}\label{thm:sigarea}
    Let $\Gamma$ has signature $(g;m_1,m_2,\cdots,m_r;s)$. Then 
    $$\mu(\H/\Gamma)=2\pi\left((2g-2)+\sum_{i=1}^r\left(1-\frac{1}{m_i}\right)+s\right)$$
    In the cocompact case where $s=0$, we have
    $$\mu(\H/\Gamma)=2\pi\left((2g-2)+\sum_{i=1}^r\left(1-\frac{1}{m_i}\right)\right)$$
\end{theorem}

\begin{proof}
    $\mu(\H/\Gamma)=\mu(F)$ and $F$ has $r$ elliptic cycles of vertices. The sum of the angles at all elliptic vertices is $\sum_{i=1}^r\frac{2\pi}{m_i}$ by \Cref{thm:2pim}. There exist $s$ parabolic cycles and the order of the stabilizer of these vertices is $\infty$, so the sum of the angles at those vertices is $\sum_{i=1}^s\frac{2\pi}{\infty}=0$. Hence, the sum of all angles of $F$ is $$2\pi\left(\sum_{i=1}^r\frac{1}{m_i}\right)$$
    The sides of $F$ are matched up by the elements of $\Gamma$. If we identify these sides, we get an orbifold of genus $g$. It has $r+s$ vertices, $1$ face, and $n$ edges, where $n$ is the number of sets of identified sides. By the Euler formula, $$2-2g=\chi=(r+s)-n+1$$

    \begin{lemma*}[Gauss-Bonnet]
        A $n$ sided hyperbolic polygon $P$ with angles $\alpha_1,\alpha_2,\cdots,\alpha_n$ has area $$\mu(P)=(n-2)\pi+\sum_{i=1}^n\alpha_i$$
        To prove it, simply divide the polygon into triangles and use $\mu(\triangle)=\pi-\alpha-\beta-\gamma$.
    \end{lemma*}

    $F$ has $2n$ sides, $2$ for every matched up set. By the Gauss-Bonnet formula, we have 
    \begin{align*}
        \mu(F) &= (2n-2)\pi-2\pi\left(\sum_{i=1}^r\frac{1}{m_i}\right)\\
        &= 2\pi\left((n-1)-\sum_{i=1}^r\frac{1}{m_i}\right)\\
        &= 2\pi\left((n-1)-(1-2g+n-r-s)-\sum_{i=1}^r\frac{1}{m_i}\right)\\
        &= 2\pi\left((2g-2)+\sum_{i=1}^r \left(1-\frac{1}{m_i}\right)+s\right)
    \end{align*}
\end{proof}

\begin{example*}
    The modular group has signature $(0;2,3;1)$, so 
    $$\mu(\H/\PSL_2(\mathbb{Z}))=2\pi\left(-2+\frac{1}{2}+\frac{2}{3}+1\right)=\frac{\pi}{3}$$
\end{example*}

Are all signatures possible? It is not, however, the only restriction is to get a positive area from the formula in \Cref{thm:sigarea}.

\begin{maintheorem}[Poincare]\label{thm:poincare}
    There exists a Fuchsian group with signature $(g;m_1,m_2,\cdots,m_r;s)$ if $g\geq 0$, $r\geq 0$, $m_i\geq 2$, $s\geq 0$ are integers and $$(2g-2)+\sum_{i=1}^r\left(1-\frac{1}{m_i}\right)+s>0$$
\end{maintheorem}

\begin{proof}
    Take a regular $4g+r+s$ sided hyperbolic polygon in unit disc model where the vertices are $0<t<1$ Euclidean distance from the center. On the first $r$ sides, construct external isosceles hyperbolic triangles with apex angle $\frac{2\pi}{m_i}$ (in case $m_i$ is $2$, we take the midpoint of the base as the apex vertex) and on the next $s$ sides, construct isosceles hyperbolic triangles with apex angle $0$ to turn it into a polygon $P(t)$ with $4g+2r+2s$ vertices. The \Cref{fig:fds} depicts an example for $g=2$, $r=3$, $s=1$.

    \newpage

    \begin{figure}
    \centering
    \begin{tikzpicture}[scale=5.0]
        \draw (0,0) circle (1);
        \clip (0,0) circle (1);
        \hgline{0}{180}
        \hgline{30}{210}
        \hgline{60}{240}
        \hgline{90}{270}
        \hgline{120}{300}
        \hgline{150}{330}
        \begin{scope}
            \clip (0,0) circle (0.7);
            \hgline{\anglex+30*0}{-\anglex+30*0-30}
            \hgline{\anglex+30*1}{-\anglex+30*1-30}
            \hgline{\anglex+30*2}{-\anglex+30*2-30}
            \hgline{\anglex+30*3}{-\anglex+30*3-30}
            \hgline{\anglex+30*4}{-\anglex+30*4-30}
            \hgline{\anglex+30*5}{-\anglex+30*5-30}
            \hgline{\anglex+30*6}{-\anglex+30*6-30}
            \hgline{\anglex+30*7}{-\anglex+30*7-30}
            \hgline{\anglex+30*8}{-\anglex+30*8-30}
            \hgline{\anglex+30*9}{-\anglex+30*9-30}
            \hgline{\anglex+30*10}{-\anglex+30*10-30}
            \hgline{\anglex+30*11}{-\anglex+30*11-30}
        \end{scope}
    
        \node[label=$v_{1}$] at ({0.7 * cos(30*1-30)}, {-0.7 * sin(30*1-30)})[circle,fill,inner sep=1.0pt]{};
        \node[label=$v_{4}$] at ({0.7 * cos(30*2-30)}, {-0.7 * sin(30*2-30)})[circle,fill,inner sep=1.0pt]{};
        \node[label=$v_{3}$] at ({0.7 * cos(30*3-30)}, {-0.7 * sin(30*3-30)})[circle,fill,inner sep=1.0pt]{};
        \node[label=$v_{2}$] at ({0.7 * cos(30*4-30)}, {-0.7 * sin(30*4-30)})[circle,fill,inner sep=1.0pt]{};
        \node[label=$v_{5}$] at ({0.7 * cos(30*5-30)}, {-0.7 * sin(30*5-30)})[circle,fill,inner sep=1.0pt]{};
        \node[label=$v_{8}$] at ({0.7 * cos(30*6-30)}, {-0.7 * sin(30*6-30)})[circle,fill,inner sep=1.0pt]{};
        \node[label=$v_{7}$] at ({0.7 * cos(30*7-30)}, {-0.7 * sin(30*7-30)})[circle,fill,inner sep=1.0pt]{};
        \node[label=$v_{6}$] at ({0.7 * cos(30*8-30)}, {-0.7 * sin(30*8-30)})[circle,fill,inner sep=1.0pt]{};
        \node[label=$v_{9}$] at ({0.7 * cos(30*9-30)}, {-0.7 * sin(30*9-30)})[circle,fill,inner sep=1.0pt]{};
        \node[label=$v_{10}$] at ({0.7 * cos(30*10-30)}, {-0.7 * sin(30*10-30)})[circle,fill,inner sep=1.0pt]{};
        \node[label=$v_{11}$] at ({0.7 * cos(30*11-30)}, {-0.7 * sin(30*11-30)})[circle,fill,inner sep=1.0pt]{};
        \node[label=$v_{12}$] at ({0.7 * cos(30*12-30)}, {-0.7 * sin(30*12-30)})[circle,fill,inner sep=1.0pt]{};
        
        \node[label=$w_1$] at ({0.8 * cos(15)}, {0.8 * sin(15)})[circle,fill,inner sep=1.0pt]{};
        \begin{scope}
            \clip (0,0) circle (0.8);
            \clip ({0.8 * cos(15)}, {0.8 * sin(15)}) circle (0.22);
            \hgline{10}{50}
            \hgline{-20}{20}
        \end{scope}
        \node[label=$w_2$] at ({0.635 * cos(45)}, {0.635 * sin(45)})[circle,fill,inner sep=1.0pt]{};
        \node[label=$w_3$] at ({0.9 * cos(75)}, {0.9 * sin(75)})[circle,fill,inner sep=1.0pt]{};
        \begin{scope}
            \clip (0,0) circle (0.9);
            \clip ({0.9 * cos(75)}, {0.9 * sin(75)}) circle (0.28);
            \hgline{74}{115}
            \hgline{35}{76}
        \end{scope}
        \node[label=$w_4$] at ({1.0 * cos(105)}, {1.0 * sin(105)})[circle,fill,inner sep=1.0pt]{};
        \begin{scope}
            \clip ({1.0 * cos(105)}, {1.0 * sin(105)}) circle (0.37);
            \hgline{105}{147}
            \hgline{63}{105}
        \end{scope}
        
        \draw[-Stealth] (-0.14+0.005,0.776-0.004) -- (-0.14,0.776) node[above]{\footnotesize $\xi_4'$};
        \draw[-Stealth] (-0.267-0.001,0.749-0.004) -- (-0.267,0.749) node[above left]{\footnotesize $\xi_4$};
    
        \draw[-Stealth] (0.14-0.005,0.767-0.004) -- (0.14,0.767) node[above]{\footnotesize $\lambda_3'$};
        \draw[-Stealth] (0.26+0.001,0.749-0.004) -- (0.26,0.749) node[right]{\footnotesize $\lambda_3$};
    
        \draw[-Stealth] (0.4-0.005,0.51+0.007) -- (0.4,0.51) node[below]{\footnotesize $\lambda_2'$};
        \draw[-Stealth] (0.51+0.007,0.4-0.005) -- (0.51,0.4) node[below]{\footnotesize $\lambda_2$};
    
        \draw[-Stealth] (0.684-0.005,0.25+0.007) -- (0.684,0.25) node[below]{\footnotesize $\lambda_1'$};
        \draw[-Stealth] (0.726-0.001,0.125-0.004) -- (0.726,0.125) node[right]{\footnotesize $\lambda_1$};
        
        \draw[-Stealth] ({0.639*cos(15+30*4)+0.01*sin(15+30*4)},{0.639*sin(15+30*4)-0.01*cos(15+30*4)}) -- ({0.639*cos(15+30*4)},{0.639*sin(15+30*4)}) node[right]{\footnotesize $\alpha_1$};
        \draw[-Stealth] ({0.639*cos(15+30*5)+0.01*sin(15+30*5)},{0.639*sin(15+30*5)-0.01*cos(15+30*5)}) -- ({0.639*cos(15+30*5)},{0.639*sin(15+30*5)}) node[right]{\footnotesize $\beta_1'$};
        \draw[-Stealth] ({0.639*cos(15+30*8)+0.01*sin(15+30*8)},{0.639*sin(15+30*8)-0.01*cos(15+30*8)}) -- ({0.639*cos(15+30*8)},{0.639*sin(15+30*8)}) node[above]{\footnotesize $\alpha_2$};
        \draw[-Stealth] ({0.639*cos(15+30*9)+0.01*sin(15+30*9)},{0.639*sin(15+30*9)-0.01*cos(15+30*9)}) -- ({0.639*cos(15+30*9)},{0.639*sin(15+30*9)}) node[above]{\footnotesize $\beta_2'$};
    
        \draw[-Stealth] ({0.639*cos(15+30*6)-0.01*sin(15+30*6)},{0.639*sin(15+30*6)+0.01*cos(15+30*6)}) -- ({0.639*cos(15+30*6)},{0.639*sin(15+30*6)}) node[right]{\footnotesize $\alpha_1'$};
        \draw[-Stealth] ({0.639*cos(15+30*7)-0.01*sin(15+30*7)},{0.639*sin(15+30*7)+0.01*cos(15+30*7)}) -- ({0.639*cos(15+30*7)},{0.639*sin(15+30*7)}) node[above]{\footnotesize $\beta_1$};
        \draw[-Stealth] ({0.639*cos(15+30*10)-0.01*sin(15+30*10)},{0.639*sin(15+30*10)+0.01*cos(15+30*10)}) -- ({0.639*cos(15+30*10)},{0.639*sin(15+30*10)}) node[right]{\footnotesize $\alpha_2'$};
        \draw[-Stealth] ({0.639*cos(15+30*11)-0.01*sin(15+30*11)},{0.639*sin(15+30*11)+0.01*cos(15+30*11)}) -- ({0.639*cos(15+30*11)},{0.639*sin(15+30*11)}) node[right]{\footnotesize $\beta_2$};
    \end{tikzpicture}
    \caption{The polygon $P(t)$}
    \label{fig:fds}
    \end{figure}
    
    As $t\rightarrow 0$, $\mu(P(t))\rightarrow 0$. As $t\rightarrow 1$, the angles except for $\frac{2\pi}{m_i}$ vanish, so by the Gauss-Bonnet formula, we have $$\mu(P(t))=(4g+2r+2s-2)\pi-\sum_{i=1}^r\frac{2\pi}{m_i}=2\pi\left((2g-1)+\sum_{i=1}^r\left(1-\frac{1}{m_i}\right)+s\right)$$
    Since $\mu(P(t))$ is continuous, for some $\tilde{t}$ between $0$ and $1$, $\mu(P(\tilde{t}))$ becomes the desired value 
    $$\mu(P(\tilde{t}))=2\pi\left((2g-2)+\sum_{i=1}^r\left(1-\frac{1}{m_i}\right)+s\right)$$
    in \Cref{thm:sigarea}. We take this $P(\tilde{t})$ as our polygon $P$.

    For any two geodesics with equal length, there exists an isometry mapping one to other. Take $A_i$, $B_j$, $X_k$, $Y_l$ for $i,j\in\{1,2,\cdots,g\}$, $k\in\{1,2,\cdots,r\}$, $l\in\{1,2,\cdots s\}$ such that
    $$A(\alpha'_i)=\alpha_i,\qquad B(\beta'_j)=\beta_j,\qquad X(\lambda'_k)=\lambda_k,\qquad Y(\xi'_l)=Y(\xi_l)$$

    Now, we compute the congruence classes of the vertices. $v_1$ is congruent to $v_2$ via $B_g^{-1}$. This $v_2$ is congruent to $v_3$ via $A_g^{-1}$. This $v_3$ is congruent to $v_4$ via $B_g$. Proceeding with this process, also considering the $r+s$ vertices that we build isosceles triangles on are also congruent via $X_k$ and $Y_l$, so we see that all vertices of the regular polygon that we begin with at the start form a congruent set. So, $$X_1X_2\cdots X_rY_1Y_2\cdots Y_sA_1B_1A_1^{-1}B_1^{-1}\cdots A_gB_gA_g^{-1}B_g^{-1}(v_1)=v_1$$
    The other vertices $w_1,w_2,\cdots,w_r$ form $r$ congruent sets each with just one element.

    Let the sum of angles at the congruent set of vertices $v_1,\cdots,v_{4g+r+s}$ be $\alpha$. Because of our choice of $P$, we have
    $$\mu(P)=2\pi\left((2g-2)+\sum_{i=1}^r\left(1-\frac{1}{m_i}\right)+s\right)$$
    Also, by the Gauss-Bonnet formula again, we have
    \begin{align*}
        \mu(P) &= (4g+2r+2s-2)\pi-\left(\alpha+\sum_{i=1}^r\frac{2\pi}{m_i}\right)\\
        &= 2\pi\left((2g-1)+\sum_{i=1}^r\left(1-\frac{1}{m_i}\right)+s\right)-\alpha
    \end{align*}
    So, $\alpha=2\pi$. Let $\Gamma$ be the group generated by $A_i$, $B_j$, $X_k$, $Y_l$. By \Cref{thm:genset}, we expect $\Gamma$ to be the group we want. The sum of angles at congruent vertices $v_1,\cdots,v_{4g+r+s}$ is $2\pi$, the angle at $w_k$ is $\frac{2\pi}{m_k}$, and the other angles are $0$. By \Cref{thm:2pim}, $P$ is a fundamental domain for $\Gamma$. $\H/\Gamma$ has $r+s+1$ congruent set of vertices, $2g+r+s$ edges, and $1$ face. By Euler formula, 
    $$2-2g=(r+s+1)-(2g+r+s)+1$$
    we see that it has genus $g$. There are $r$ elliptic cycles, $\{w_1\},\{w_2\},\cdots,\{w_r\}$, and their stabilizers have orders $m_1,m_2,\cdots,m_r$. There are $s$ conjugacy classes of maximal parabolic cyclic subgroups. Hence, $\Gamma$ has signature $(g;m_1,m_2,\cdots,m_r;s)$.
\end{proof}

\begin{remark*}
    The representation of the group $\Gamma$ with signature $(g;m_1,m_2,\cdots,m_r;s)$ is 
    \begin{align*}
    \Gamma=\langle &A_1,\cdots,A_g,B_1,\cdots,B_g,X_1,\cdots,X_r,Y_1,\cdots,Y_s:\\
    &X_1^{m_1}=X_2^{m_2}=\cdots=X_r^{m_r}=\text{Id},\\
    &X_1X_2\cdots X_rY_1Y_2\cdots Y_sA_1B_1A_1^{-1}B_1^{-1}\cdots A_gB_gA_g^{-1}B_g^{-1}=\text{Id}\rangle
    \end{align*}
    because $X_k$ fixes the point $w_k$ of order $m_k$ and the stabilizer of $v_1$ is trivial.
\end{remark*}

\begin{figure}
    \centering
    \begin{tabular}{ccc}
    
    \includegraphics[width=0.25\textwidth]{figures/hy/2-3-7-blue-v2.png} &
    \includegraphics[width=0.25\textwidth]{figures/hy/2-3-8-blue.png} & 
    \includegraphics[width=0.25\textwidth]{figures/hy/2-3-i-black.png} \\
    $(2,3,7)$ & $(2,3,8)$ & $(2,3,\infty)$ \\[6pt]
    
    \includegraphics[width=0.25\textwidth]{figures/hy/2-4-5-black.png} &
    \includegraphics[width=0.25\textwidth]{figures/hy/2-4-6-red.png} & 
    \includegraphics[width=0.25\textwidth]{figures/hy/2-4-8-black.png} \\
    $(2,4,5)$ & $(2,4,6)$ & $(2,4,8)$ \\[6pt]

    \includegraphics[width=0.25\textwidth]{figures/hy/2-5-5-black.png} &
    \includegraphics[width=0.25\textwidth]{figures/hy/2-5-6-black.png} & 
    \includegraphics[width=0.25\textwidth]{figures/hy/2-5-i-black.png} \\
    $(2,5,5)$ & $(2,5,6)$ & $(2,5,\infty)$ \\[6pt]

    \includegraphics[width=0.25\textwidth]{figures/hy/2-6-6-red.png} &
    \includegraphics[width=0.25\textwidth]{figures/hy/2-6-8-black.png} & 
    \includegraphics[width=0.25\textwidth]{figures/hy/2-i-i-black.png} \\
    $(2,6,6)$ & $(2,6,8)$ & $(2,\infty,\infty)$ \\[6pt]

    \includegraphics[width=0.25\textwidth]{figures/hy/3-3-4-blue.png} &
    \includegraphics[width=0.25\textwidth]{figures/hy/5-i-i-black.png} & 
    \includegraphics[width=0.25\textwidth]{figures/hy/i-i-i-black.png} \\
    $(3,3,4)$ & $(5,\infty,\infty)$ & $(\infty,\infty,\infty)$ \\[6pt]
    
    \end{tabular}
    \caption{Some examples of triangle groups}
    \label{fig:tri}
\end{figure}

\begin{example*}
    The minimum area possible for a non-cocompact Fuchsian group is attained by the modular group $(0;2,3;1)=(0;2,3,\infty)$
    $$\mu(\H/(0;2,3;1))=\frac{\pi}{3}$$

    The minimum area possible for a cocompact Fuchsian group is attained by $(0;2,3,7)$
    $$\mu(\H/(0;2,3,7))=\frac{\pi}{21}$$
\end{example*}

\begin{corollary*}[triangle group]
    A Fuchsian group with the signature of the form $(0;m_1,\cdots,m_r;s)$ where $r+s=3$ is called a triangle group. We can also denote its signature with $(0;m_1,m_2,m_3)$ allowing $m_i$ to be infinity. By \Cref{thm:poincare}, a triangle group exists if and only if $$\frac{1}{m_1}+\frac{1}{m_2}+\frac{1}{m_3}<1$$
\end{corollary*}

Some interesting examples of triangle groups can be found in \Cref{fig:tri}. The figure below shows how a triangle group can be constructed by reflecting a hyperbolic triangle with angles $\frac{\pi}{m_1}$, $\frac{\pi}{m_2}$, $\frac{\pi}{m_3}$ over its sides.

\begin{center}
    \includegraphics[scale=0.3]{figures/reflection-triangle.png}
\end{center}

% https://en.wikipedia.org/wiki/(2,3,7)_triangle_group

\begin{thebibliography}{9}
    \bibitem{ref:katok} \textbf{Fuchsian Groups}, by Svetlana Katok.
\end{thebibliography}


\end{document}
