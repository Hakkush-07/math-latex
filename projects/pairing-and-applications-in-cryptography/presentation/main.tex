\documentclass[
	american,
	sections numbered,
	usenames,
	xcolor=dvipsnames,
	aspectratio=1610,
	%hyperref={pdfpagemode=FullScreen},
]{beamer}

\mode<presentation>

\usepackage[T1]{fontenc}
\usepackage{FiraMono}

\usetheme[progressbar=frametitle]{metropolis}

\usepackage{ifthen}

%%% GRAPHICS %%%
\usepackage{graphicx}
\usepackage{pgfplots}
\usepgfplotslibrary{polar}
\usepackage{tikz}
\usetikzlibrary{arrows.meta}

%%% MATH & SCIENCE %%%
\usepackage{amsmath}
\usepackage{amssymb}
\usepackage{amsfonts}
\usepackage{amsthm}
\usepackage{siunitx}
\usepackage{bm}
\usepackage{dsfont}
\usepackage{mathtools}

%%% FLOATS %%%
\usepackage{booktabs}
\usepackage{tabularx}

%\usepackage{biblatex}
%\bibliography{literature.bib}


% DESIGN COLORS
\definecolor{accent}{HTML}{7EBDC2} % accent color
\definecolor{bgcolor}{HTML}{FCFCFF} % background color
\definecolor{bgcolorAlt}{HTML}{DCE1FC} % alternative background color
\definecolor{fgcolor}{HTML}{222244} % foreground/text color

%
\setbeamercolor{normal text}{%
	fg=fgcolor,
	bg=bgcolor,
}
\setbeamercolor{alerted text}{%
	fg=accent,
}
\setbeamercolor{palette primary}{%
	use=normal text,
	fg=normal text.fg,
	bg=bgcolorAlt,%normal text.bg
}

\setbeamercolor{block title}{
	bg=bgcolorAlt,
}
\setbeamercolor{block body}{
	bg=bgcolorAlt,
}
\setbeamercolor{block title alerted}{%
	use={palette primary, alerted text},
	fg=palette primary.bg,
	bg=alerted text.fg
}
\setbeamercolor{block title example}{%
	use={block title, alerted text},
	bg=block title.bg,
	fg=alerted text.fg
}
%

\pgfplotsset{legend style={fill=bgcolor,draw=fgcolor}}

% PLOT COLORS
%% Paul Tol High Contrast
\definecolor{plot0}{HTML}{004488}
\definecolor{plot1}{HTML}{DDAA33}
\definecolor{plot2}{HTML}{BB5566}
\definecolor{plot3}{HTML}{000000}
\definecolor{plot4}{HTML}{AAAAAA}

%% Paul Tol Vibrant
%\definecolor{plot0}{HTML}{EE7733}
%\definecolor{plot1}{HTML}{0077BB}
%\definecolor{plot2}{HTML}{33BBEE}
%\definecolor{plot3}{HTML}{EE3377}
%\definecolor{plot4}{HTML}{CC3311}
%\definecolor{plot5}{HTML}{009988}
%\definecolor{plot6}{HTML}{BBBBBB}

\pgfplotscreateplotcyclelist{lineplot cycle}{ %
	{plot0, mark=*, thick, mark options=solid},
	{plot1, mark=triangle*, thick, mark options=solid},
	{plot2, mark=square*, thick, mark options=solid},
	{plot3, mark=diamond*, thick, mark options=solid},
	{plot4, mark=pentagon*, thick, mark options=solid},
}

% \AtBeginEnvironment{thm}{%
%   % \setbeamercolor{block title}{use=example text,fg=white,bg=example text.fg!75!black}
%   \setbeamercolor{block body}{parent=normal text,use=block title example,bg=blue!75!black!10!}
% }


%\renewcommand*{\bibfont}{\scriptsize}
\setbeamerfont{block body reference}{size=\scriptsize}
\setbeamerfont{block title reference}{size=\scriptsize}

\setbeamerfont{description item}{series=\mdseries}
\setbeamerfont{alerted text}{series=\bfseries\boldmath}


\setbeamertemplate{title page}{
\begin{minipage}[b][\textheight]{\textwidth}
	\ifx\inserttitlegraphic\@empty\else\usebeamertemplate*{title graphic}\fi
	\vfill%
	\ifx\inserttitle\@empty\else\usebeamertemplate*{title}\fi
	\ifx\insertsubtitle\@empty\else\usebeamertemplate*{subtitle}\fi
	\usebeamertemplate*{title separator}

	\ifx\beamer@shortauthor\@empty\else\usebeamertemplate*{author}\fi
	\ifx\insertdate\@empty\else\usebeamertemplate*{date}\fi
	\ifx\insertinstitute\@empty\else\usebeamertemplate*{institute}\fi
	\vfil
	\vspace*{1mm}
\end{minipage}
}
\newcommand*{\seprule}{{\par\color{bgcolorAlt!90!fgcolor}\hrulefill\par\vspace*{1ex plus 0pt minus .5ex}}}


% Mathematical Writing
\DeclarePairedDelimiter{\abs}{\vert}{\vert}
\DeclarePairedDelimiter{\norm}{\Vert}{\Vert}
\DeclarePairedDelimiter{\ceil}{\lceil}{\rceil}
\DeclarePairedDelimiter{\floor}{\lfloor}{\rfloor}

\newcommand*{\inv}[1]{\ensuremath{#1^{-1}}}
\newcommand*{\positive}[1]{\ensuremath{\left[#1\right]^{+}}}

\newcommand*{\diff}{\ensuremath{\mathrm{d}}}
\newcommand*{\imag}{\ensuremath{\mathrm{j}}}
\newcommand*{\e}{\ensuremath{\mathrm{e}}}

\DeclareMathOperator*{\argmax}{arg\,max}
\DeclareMathOperator*{\argmin}{arg\,min}

%% change these to \mathbb, if you do not want to use the dsfont package
\newcommand*{\reals}{\ensuremath{\mathds{R}}} 
\newcommand*{\complexes}{\ensuremath{\mathds{C}}}
\newcommand*{\naturals}{\ensuremath{\mathds{N}}}
%%

\newcommand*{\expect}[2][]{\ensuremath{\mathbb{E}_{#1}\left[#2\right]}}

\newcommand*{\unif}{\ensuremath{\mathcal{U}}}
\newcommand*{\normaldist}{\ensuremath{\mathcal{N}}}


\newcommand{\mbx}[1]{\makebox[5cm]{#1\hfill}}
\newcommand{\poly}[2]{#1_0+#1_1x+\cdots+#1_#2x^#2}
\newcommand{\polyh}[2]{#1_0(y,z)+#1_1(y,z)x+\cdots+#1_#2(y,z)x^#2}

\newcommand{\curveC}{\mathcal C}
\newcommand{\curveD}{\mathcal D}
\newcommand{\resPQ}{\mathcal R_{P,Q}}
\newcommand{\CNZ}{\CC^{n+1}-\{\vec{0}\}}
\newcommand{\projective}{\mathbb P}
\DeclareMathOperator{\Div}{Div}
\DeclareMathOperator{\Res}{Res}
\DeclareMathOperator{\ord}{ord}

\newcommand{\CC}{\mathbb C}

\newcommand{\vocab}[1]{\textbf{\color{blue}\sffamily #1}}

\newcommand{\hgline}[2]{
\pgfmathsetmacro{\thetaone}{#1}
\pgfmathsetmacro{\thetatwo}{#2}
\pgfmathsetmacro{\theta}{(\thetaone+\thetatwo)/2}
\pgfmathsetmacro{\phi}{abs(\thetaone-\thetatwo)/2}
\pgfmathsetmacro{\close}{less(abs(\phi-90),0.0001)}
\ifdim \close pt = 1pt
    \draw[blue] (\theta+180:1) -- (\theta:1);
\else
    \pgfmathsetmacro{\R}{tan(\phi)}
    \pgfmathsetmacro{\distance}{sqrt(1+\R^2)}
    \draw[blue] (\theta:\distance) circle (\R);
\fi
}

\newcommand\anglex{10}

\renewcommand{\H}{\mathbb{H}}
\DeclareMathOperator{\PSL}{PSL}

% THEOREMS
\theoremstyle{plain}
% \newtheorem{theorem}{Theorem}%[section]
% \newtheorem{lemma}{Lemma}
% \newtheorem{proposition}{Proposition}
% \newtheorem{corollary}{Corollary}
\newtheorem{factx}[theorem]{Fact}
\newtheorem{observation}[theorem]{Observation}
\newtheorem{remark}[theorem]{Remark}
% \newtheorem{example}{Example}

\setbeamertemplate{theorems}[numbered]

\pgfplotsset{compat=newest}
\pgfplotsset{%
	betterplot/.style={
		width=.93\linewidth,
		height=.5\textheight,
		xlabel near ticks,
		ylabel near ticks,
		cycle list name=lineplot cycle,
		mark options=solid,
		xmajorgrids=true,
		xminorgrids=true,
		ymajorgrids=true,
		grid style={line width=.1pt, draw=gray!20},
		major grid style={line width=.25pt,draw=gray!30},
		legend cell align=left,
		legend style = {
			/tikz/every even column/.append style={column sep=0.33cm}
		},
	},
}

\title{Pairings and Applications in Cryptography}
\author{Hakan Karakuş - Graduate Seminar}
\date{5 November 2025}

% \titlegraphic{\includegraphics[width=0.3\textheight]{figures/logo.png}}

\begin{document}
\begin{frame}[plain]
	\titlepage
\end{frame}

\begin{frame}{Table of contents}
	\setbeamertemplate{section in toc}[sections numbered]
	\tableofcontents%[hideallsubsections]
\end{frame}

\section{Pairing}

\subsection{Definition and Simple Examples}

\begin{frame}{Pairing - Definition (for Cryptography)}

    Let $G_1$ and $G_2$ be additive, and $G_T$ be multiplicative abelian groups. A \vocab{pairing} is a map $e:G_1\times G_2\rightarrow G_T$ such that

    \begin{itemize}
        \item (\textbf{bilinear}) for any $a,a_1,a_2\in G_1$, $b,b_1,b_2\in G_2$,
		\begin{itemize}
			\item $e(a_1+a_2,b)=e(a_1,b)e(a_2,b)$
			\item $e(a,b_1+b_2)=e(a,b_1)e(a,b_2)$
		\end{itemize}
        \item (\textbf{non-degenerate})
        \begin{itemize}
			\item for any $a\neq{Id}_{G_1}$, there is at least one $b\in G_2$ such that $e(a,b)\neq {Id}_{G_T}$
			\item for any $b\neq{Id}_{G_2}$, there is at least one $a\in G_1$ such that $e(a,b)\neq {Id}_{G_T}$
		\end{itemize}
		\item {\color{gray} efficiently computable}
    \end{itemize}

\end{frame}

\begin{frame}{Pairing - Simple Examples}

	\begin{itemize}
		\item Multiplication, $\bullet:\mathbb{R}\times\mathbb{R}\rightarrow\mathbb{R}$, $(x,y)\mapsto xy$
		\item Scalar product, $\langle\cdot,\cdot\rangle:\mathbb{R}^n\times\mathbb{R}^n\rightarrow\mathbb{R}$
		\item Determinant map, $\det:\mathbb{R}^2\times\mathbb{R}^2\rightarrow\mathbb{R}$, $([a, b]^T, [c, d]^T)\mapsto ad-bc$
	\end{itemize}

	These mappings can be composed with $\exp:\mathbb{R}^+\rightarrow\mathbb{R}^\times$ to get examples of pairings.

\end{frame}

\begin{frame}{Pairing - For Cyclic Groups}

	For cyclic groups $G_1=\langle P\rangle$, $G_2=\langle Q\rangle$, the pairing definition gets simplified to:

	\begin{itemize}
		\item $e(aP,bQ)=e(P,Q)^{ab}$ (bilinearity)
		\item $e(P,Q)\neq {Id}_{G_T}$ (non-degeneracy)
		\item {\color{gray} efficiently computable}
	\end{itemize}

\end{frame}

\subsection{Computational Perspective}

\begin{frame}{A bit about the Computational Perspective}

	Let $G=\langle g\rangle$ be a cyclic multiplicative group of prime order $p$. Let $0\leq x,y,z\leq p-1$ be integers.

	\textbf{Exponentiation Problem:} Given {\color{red}$g$} and {\color{red}$x$}, compute {\color{blue}$g^x$}.
	
	\textbf{Discrete Log Problem:} Given {\color{red}$g$} and {\color{red}$g^x$}, compute {\color{blue}$x$}.

	\textbf{Computational Diffie-Hellman Problem:} Given {\color{red}$g$}, {\color{red}$g^x$}, and {\color{red}$g^y$}, compute {\color{blue}$g^{xy}$}.

	\textbf{Decisional Diffie-Hellman Problem:} Given {\color{red}$g$}, {\color{red}$g^x$}, {\color{red}$g^y$}, {\color{red}$g^z$}, determine if {\color{blue}$z\overset{?}{=}xy$} or not.

\end{frame}

\begin{frame}{Difficulty of Solving These Problems}

	\vocab{hard:} there is no efficient method, try all possibilities by brute force

	\vocab{easy:} there is an efficient method

	\vocab{efficient:} the method terminates after a very small number of steps compared to the size of the input

\end{frame}

\begin{frame}{Exponentiation Problem}

	\textbf{Exponentiation Problem} is \underline{easy}. We can just use square-and-multiply method. For example, to find $a^{23}$ when $a$ is given, we need 7 multiplication operations: 
	$$a^2=a\times a$$
	$$a^4=a^2\times a^2$$
	$$a^5=a^4\times a$$
	$$a^{10}=a^5\times a^5$$
	$$a^{11}=a^{10}\times a$$
	$$a^{22}=a^{11}\times a^{11}$$
	$$a^{23}=a^{22}\times a$$

\end{frame}

\begin{frame}{Exponentiation Problem and Discrete Log Problem}

	In general case, we can find $a^x$ in just $\lfloor\log_2x\rfloor$ squaring, and possibly another $\lfloor\log_2x\rfloor$ multiplication. And crucially
	$$2\log_2x<<<x$$

	For example, for a typical group used in cryptography of order around $2^{256}$, we could have inputs possibly greater than the number of atoms in the universe, but we can calculate the exponentiation in just around 500 steps.

	However, in general, there has not been any efficient method for the inverse of this problem, namely \textbf{Discrete Log Problem}, we need to try every $0\leq x\leq p-1$. So, it is a \underline{hard} problem.

\end{frame}

\begin{frame}{Diffie-Hellman Problem}

	\textbf{Computational Diffie-Hellman Problem:} Given {\color{red}$g$}, {\color{red}$g^x$}, and {\color{red}$g^y$}, compute {\color{blue}$g^{xy}$}.

	\textbf{Decisional Diffie-Hellman Problem:} Given {\color{red}$g$}, {\color{red}$g^x$}, {\color{red}$g^y$}, {\color{red}$g^z$}, determine if {\color{blue}$z\overset{?}{=}xy$} or not.

	\vspace{1cm}

	\textbf{Computational Diffie-Hellman Problem} is similarly \underline{hard}. However, \textbf{Decisional Diffie-Hellman Problem} is \underline{easy} in groups of prime order where we can do pairings. Assume there is a symmetric pairing $e:G\times G\rightarrow G_T$. We just check the following equality.

	$$e(g,g)^z=e(g,g^z)\overset{?}{=}e(g^x,g^y)=e(g,g)^{xy}$$

\end{frame}

\section{Elliptic Curves}

\begin{frame}{Elliptic Curves}

	An \vocab{elliptic curve} $E$ over a field $K$ is an equation of the form 
	$$y^2=x^3+ax+b$$
	such that $\text{char}\,K\not\in \{2,3\}$ (we need this so that we can convert them into this nice form) and $4a^3+27b^2\neq 0$ (we need this so that the curve is non-singular i.e. has no cusps or self intersections, partial derivatives do not vanish simultaneously).
	
	This is called the short Weierstrass form.
	$$E(K)=\{(x,y)\in K\times K: y^2=x^3+ax+b\}\cup\{\mathcal{O}\}$$

	$E(K)$ has a nice abelian group structure with $\mathcal{O}$ being the identity (Mordell-Weil Theorem).

\end{frame}

\subsection{Group Operation}

\begin{frame}{Elliptic Curves - Group Operation}

	\begin{minipage}{0.4\textwidth}
		Let $P$ and $Q$ be points on $E(K)$. The line (degree 1 curve) passing through $P$ and $Q$ intersects $E(K)$ again at one more point due to Bezout Theorem (it can also be $P$ or $Q$ or $\mathcal{O}$). Reflecting this point over the $x$ axis, we get the point that we define to be $P+Q$.

		The line through $P$ and $\mathcal{O}$ means the vertical line through $P$. The line through $P$ and $P$ means the tangent at $P$.
	\end{minipage}
	\begin{minipage}{0.59\textwidth}
		\begin{figure}
			\centering
			\begin{tikzpicture}[scale=1.1]
				\draw[->] (-2.3, 0) -- (3.4, 0) node[right] {$x$};
				\draw[->] (0, -3.0) -- (0, 3.0) node[above] {$y$};
				\draw[domain=-1.987076:3.1,samples=100,smooth,variable=\x,blue] plot ({\x},{sqrt((\x^3)/3.375 - (\x)/1.5 + 1)});
				\draw[domain=-1.987076:3.1,samples=100,smooth,variable=\x,blue] plot ({\x},{-sqrt((\x^3)/3.375 - (\x)/1.5 + 1)});
				\node (P) at (-1.95, 0.321) {};
				\node (Q) at (0.6, 0.81486) {};
				\node (NPQ) at (1.47657, 0.98463) {};
				\node (PQ) at (1.47657, -0.98463) {};
				\filldraw[black] (P) circle (1pt) node[anchor=east]{$P$};
				\filldraw[black] (Q) circle (1pt) node[anchor=south]{$Q$};
				\filldraw[black] (NPQ) circle (1pt) node[anchor=west]{$-(P+Q)$};
				\filldraw[black] (PQ) circle (1pt) node[anchor=west]{$P+Q$};
				\draw[dashed] (P) -- (NPQ);
				\draw[dashed, color=orange] (NPQ) -- (PQ);
			\end{tikzpicture}
			\caption{Elliptic curve group operation}
			\label{fig:ec}
		\end{figure}
	\end{minipage}

\end{frame}

\begin{frame}{Elliptic Curves - Associativity}

	\begin{minipage}{0.24\textwidth}
		\vspace{-1cm}
		Except associativity, all other properties that we care about are trivial. We can prove associativity by doing all the line curve intersection calculations, or using more sophisticated theorems like Riemann-Roch or Cayley-Bacharach {\color{blue}\cite{ref:silverman}(Silverman)}.
	\end{minipage}
	\begin{minipage}{0.75\textwidth}
		\begin{figure}
			\centering
			\begin{tikzpicture}[scale=1.2]
				\draw[->] (-2.3, 0) -- (3.4, 0) node[right] {$x$};
				\draw[->] (0, -3.0) -- (0, 3.0) node[above] {$y$};
				\draw[domain=-1.987076:3.1,samples=100,smooth,variable=\x,blue] plot ({\x},{sqrt((\x^3)/3.375 - (\x)/1.5 + 1)});
				\draw[domain=-1.987076:3.1,samples=100,smooth,variable=\x,blue] plot ({\x},{-sqrt((\x^3)/3.375 - (\x)/1.5 + 1)});
				\node (P) at (-1.95, 0.321) {};
				\node (Q) at (0.6, 0.81486) {};
				\node (R) at (0.16233, -0.945) {};
				\node (NPR) at (3.0, -2.64575) {};
				\node (PR) at (3.0, 2.64575) {};
				\node (NPQ) at (1.47657, 0.98463) {};
				\node (PQ) at (1.47657, -0.98463) {};
				\node (NPQR) at (-1.63584, -0.8908) {};
				\filldraw[black] (P) circle (1pt) node[anchor=east]{$P$};
				\filldraw[black] (Q) circle (1pt) node[anchor=south]{$Q$};
				\filldraw[black] (R) circle (1pt) node[anchor=north]{$R$};
				\filldraw[black] (NPR) circle (1pt) node[anchor=west]{$-(P+R)$};
				\filldraw[black] (PR) circle (1pt) node[anchor=west]{$P+R$};
				\filldraw[black] (NPQ) circle (1pt) node[anchor=west]{$-(P+Q)$};
				\filldraw[black] (PQ) circle (1pt) node[anchor=west]{$P+Q$};
				\filldraw[black] (NPQR) circle (1pt) node[anchor=east]{$-(P+Q+R)$};
				\draw[dashed] (P) -- (NPQ);
				\draw[dashed] (P) -- (NPR);
				\draw[dashed, color=orange] (NPQ) -- (PQ);
				\draw[dashed, color=orange] (NPR) -- (PR);
				\draw[dashed] (NPQR) -- (PQ);
				\draw[dashed] (NPQR) -- (PR);
			\end{tikzpicture}
			\caption{Associativity of the elliptic curve group operation}
			\label{fig:ec}
		\end{figure}
	\end{minipage}

\end{frame}

\begin{frame}{Elliptic Curves - Simple Example}

	\def\nn{11}
	\def\aa{9}
	\def\bb{0}

	\begin{figure}
		\centering
		\begin{tikzpicture}[scale=0.5]
			\draw[-latex] (-0.5,0) -- (\nn+0.5,0);
			\foreach \x in {0,...,\nn} {
				\ifthenelse{\x=\nn}{}{
					\draw[shift={(\x,0)},color=black] (0pt,3pt) -- (0pt,-3pt);
					\draw[shift={(\x,0)},color=black] (0pt,0pt) -- (0pt,-3pt) node[below] {\tiny $\x$};
				}
				\draw[shift={(\x,0)},dotted,color=darkgray] (0,0) -- (0,\nn);
			}
			\draw[-latex] (0,-0.5) -- (0,\nn+0.5);
			\foreach \y in {0,...,\nn} {
				\ifthenelse{\y=\nn}{}{
					\draw[shift={(0,\y)},color=black] (3pt,0pt) -- (-3pt,0pt);
					\draw[shift={(0,\y)},color=black] (0pt,0pt) -- (-3pt,0pt) node[left] {\tiny $\y$};
				}
				\draw[shift={(0,\y)},dotted,color=darkgray] (0,0) -- (\nn,0);
			}

			\foreach \x in {0,...,\nn} {
				\foreach \y in {0,...,\nn} {
					\pgfmathparse{int(mod(\y * \y - \x * \x * \x - \aa * \x - \bb,\nn))}
					\ifthenelse{\pgfmathresult=0}{
						\ifthenelse{\x<\nn}{
							\ifthenelse{\y<\nn}{
								\fill (\x, \y) circle (4pt);
							}{}
						}{}
					}{}
				}
			}

			\draw[blue, line width=1pt] (2, 9) -- (6, 11);
			\draw[blue, line width=1pt] (6, 0) -- (8, 1);
			\draw[cyan, line width=0.6pt] (8, 1) -- (8, 10);
			\fill[red] (2, 9) circle (5pt);
			\fill[red] (4, 10) circle (5pt);
			\fill[green] (8, 1) circle (4pt);
			\fill[red] (8, 10) circle (5pt);
		\end{tikzpicture}
		\caption{$y^2=x^3+9x$ over $\mathbb{F}_{11}$, showing $(2, 9)+(4, 10)=(8, 10)$}
		\label{fig:ec2}
	\end{figure}

\end{frame}

\begin{frame}{Elliptic Curves - Projective Coordinates}

	To properly represent the point at infinity, we sometimes switch to projective coordinates. With the equivalence relation $(x_1,y_1,z_1)\sim(x_2,y_2,z_2)$ iff $\exists\lambda$, $(x_1,y_1,z_1)=(\lambda x_2,\lambda y_2,\lambda z_2)$, we denote the projective points by the equivalence classes $(X:Y:Z)\in (K\times K\times K)/\sim$. We have the trivial inclusion map $(x,y)\hookrightarrow (x:y:1)$ and back mapping 
	$$(X:Y:Z)=\begin{cases}
		\left(\frac{X}{Z},\frac{Y}{Z}\right),   & \text{if }Z\neq 0 \\
		\mathcal{O},                            & (0:1:0)
	\end{cases}$$
	
	Making this substitution to the elliptic curve equation $y^2=x^3+ax+b$, we get
	$$Y^2Z=X^3+aXZ^2+bZ^3$$

\end{frame}

\begin{frame}{Elliptic Curve Cryptography - Torsion Points}

	Let $K=\mathbb{F}_q$ and let $E$ be an elliptic curve defined over $K$.

	We denote the set of $r$-torsion points in $E(K)$ by 
	$$E(K)[r]=\{P: rP=\mathcal{O}\}$$

	For this definition to be meaningful, we use $r$ such that $r{\big|}\,|E(K)|$.

	We denote all torsion points in $\overline{K}$ by $E[r]=E(\overline{K})[r]$.

\end{frame}

\begin{frame}{Elliptic Curve Cryptography - Embedding Degree}

	The following are equivalent:
	\begin{itemize}
		\item $k$ is the smallest integer such that $r\mid q^k-1$
		\item $k$ is the smallest integer such that $\mu_r\subset \mathbb{F}_{q^k}^\times$
		\item $k$ is the smallest integer such that $E[r]\subset E(\mathbb{F}_{q^k})$
	\end{itemize}
	This $k$ is called \vocab{embedding degree}.

	From now on, we denote by $E$ the curve defined over that extension field $\mathbb{F}_{q^k}$, which is $E(\mathbb{F}_{q^k})$. So, it contains all $r$-torsion points.

\end{frame}

\begin{frame}{Elliptic Curve Cryptography - Embedding Degree}

	\begin{minipage}{0.54\textwidth}
		\textbf{Example:} Consider $y^2=x^3+4$ over $\mathbb{F}_{11}$. $|E(\mathbb{F}_q)|=12$, so take $r=3$. $$E(\mathbb{F}_{q})[r]=\{\mathcal{O},(0,2),(0,9)\}$$
		Form $\mathbb{F}_{q^2}=\mathbb{F}_q/(i^2+1)$. Then $E(\mathbb{F}_{q^2})[r]$ is
	\end{minipage}
	\begin{minipage}{0.45\textwidth}
		\begin{figure}
			\centering
			\begin{tikzpicture}
				\begin{polaraxis}[grid=none, axis lines=none]
					\addplot[mark=none,domain=0:360,samples=300] {cos(2*x)};
					\draw (0,0) node[circle,minimum size=0.5cm,fill=white,draw] {$\mathcal{O}$};
					\draw (40,0) node {\tiny $(0, 2)$};
					\draw (80,0) node {\tiny $(0, 9)$};
					\draw (-40,0) node {\tiny $(8, i)$};
					\draw (-80,0) node {\tiny $(8, 10i)$};
					\draw (0,40) node {\tiny $(2i+7, i)$};
					\draw (0,80) node {\tiny $(2i+7, 10i)$};
					\draw (0,-40) node {\tiny $(9i+7, i)$};
					\draw (0,-80) node {\tiny $(9i+7, 10i)$};
				\end{polaraxis}
			\end{tikzpicture}
			\caption{$E[r]$}
			\label{fig:ec3}
		\end{figure}
	\end{minipage}

\end{frame}

\subsection{Functions on Elliptic Curves and Divisors}

\begin{frame}{Elliptic Curve Cryptography - Rational Functions on Elliptic Curves}

	Let $K[E]=K[X,Y]/E$ be the polynomials of $X$ and $Y$ such that if they agree on $E$, then they are considered equivalent. And let $K(E)$ be the field of fractions of $K[E]$.

\end{frame}

\begin{frame}{Elliptic Curve Cryptography - Divisors}

	A \vocab{divisor} $D$ on an elliptic curve $E$ is a formal sum
	$$D=\sum_{P\in E}n_P(P)$$
	such that all but finitely many $n_P\in\mathbb{Z}$ are zero.

	The \vocab{support} of $D$ is $$\text{supp}(D)=\{P:n_P\neq 0\}$$
	The \vocab{degree} of $D$ is $$\deg(D)=\sum_{P\in E}n_P$$

	For example, $2(P)+3(Q)-(R)$ is a divisor with degree 4 and support $\{P,Q,R\}$.

\end{frame}

\begin{frame}{Elliptic Curve Cryptography - Divisor of a Function}

	The divisor of a rational function $f\in K(E)$ on $E$ is
	$$(f)=\sum_{P\in E}\ord_P(f)(P)$$
    where $\ord_P(f)$ is the multiplicity of $f$ at $P$ (positive when $P$ is a zero of $f$ and negative when $P$ is a pole).

	Note that for functions $f$ and $g$, we have
	$$(fg)=(f)+(g)$$

\end{frame}

\begin{frame}{Elliptic Curve Cryptography - Examples}

	The chord line through $P$ and $Q$ has divisor
	$$(l_{P,Q})=(P)+(Q)+(-(P+Q))-3(\mathcal{O})$$

	The vertical line through $P$ (also passes through $-P$) has divisor
	$$(v_{P})=(P)+(-P)-2(\mathcal{O})$$

	The reason for $-3(\mathcal{O})$ appearing there is that if we substitute the projective version of a line $y=mx+n$ which is $\frac{Y}{Z}=\frac{mX+nZ}{Z}$ into the projective curve equation, we get 
	$$-\left(\frac{mX+nZ}{Z}\right)^2+\left(\frac{X}{Z}\right)^3+a\left(\frac{X}{Z}\right)+b$$
	which has a pole of degree $3$ when $Z=0$.

\end{frame}

\begin{frame}{Elliptic Curve Cryptography - Principal Divisor}

	If $D$ is a divisor of a function, we call it \vocab{principal}. A divisor is principal if and only if both of the following hold:
	$$\sum_{P\in E}n_P=\deg(D)=0$$
	$$\sum_{P\in E}n_PP=\mathcal{O}$$
	For a detailed proof, you can see {\color{blue}\cite{ref:galbraith}(Galbraith, Thm 7.7.11)}.

\end{frame}

\begin{frame}{Elliptic Curve Cryptography - Principal Divisor Example}

	\textbf{Example:} Consider $y^2=x^3+20x+20$ over $\mathbb{F}_{103}$. 
	
	$P=(26,20)$, $Q=(63,78)$, $R=(59,95)$, $S=(77,84)$ all on $E(\mathbb{F}_{103})$. 
	
	The divisor $(P)+(Q)-(R)-(S)$ is principal because $1+1-1-1=0$ and $P+Q-R-S=\mathcal{O}$. 
	
	Therefore, there exists a function $f$ on $E$ such that $(f)=(P)+(Q)-(R)-(S)$. 
	
	For example, $$f=\frac{6y+71x^2+91x+91}{x^2+70x+11}$$
	works.

\end{frame}

\begin{frame}{Elliptic Curve Cryptography - Evaluating a Function at a Divisor}

	For a function $f$ and a divisor $\displaystyle D=\sum_{P\in E}n_P(P)$ such that $(f)$ and $D$ have disjoint supports, we define the \vocab{evaluation of $f$ at $D$} by:
	$$f(D)=\prod_{P\in E}f(P)^{n_P}$$

\end{frame}

\begin{frame}{Elliptic Curve Cryptography - Evaluating a Function at a Divisor}

	\textbf{Example:} Consider $y^2=x^3-x-2$ over $\mathbb{F}_{163}$. 
	
	Let $P=(43, 154)$, $Q=(46, 38)$ be points on the curve. 
	
	The line passing through $P$ and $Q$ is $l:y+93x+85$. 
	
	Let $D=2(R)+(S)$ where $R=(12, 35)$ and $S=(5, 66)$. 
	
	Then 
	$$l(D)=l(R)^2l(S)=(y_R+93x_R+85)^2(y_S+93x_S+85)=122$$

\end{frame}

% \begin{frame}{Elliptic Curve Cryptography - Weil Reciprocity}

% 	For functions $f$ and $g$ on $E$ such that $(f)$ and $(g)$ have disjoint supports, we have $$f((g))=g((f))$$
% 	For a detailed proof, you can see {\color{blue}\cite{ref:galbraith}(Galbraith, Thm 26.1.2)}.

% \end{frame}

\section{Weil Pairing}

\begin{frame}{Weil Pairing - Miller Functions}

	For any $m\in\mathbb{Z}^+$ and $P\in E$, we have a function $f_{m,P}$ such that
	$$(f_{m,P})=m(P)-(mP)-(m-1)(\mathcal{O})$$
	
	For any $P\in E[r]$, we have a function $f_{r,P}$ such that 
	$$(f_{r,P})=r(P)-(rP)-(r-1)(\mathcal{O})=r(P)-r(\mathcal{O})$$

\end{frame}

\begin{frame}{Weil Pairing - Efficient Construction of Miller Functions}

	We can iteratively construct $f_{m,P}$ for big $m$ in logarithmic time. $f_{1,P}$ is just constant function, and we observe that
	\begin{align*}
		(f_{m+n,P})-(f_{m,P})-(f_{n,P}) &= (mP)+(nP)-((m+n)P)-(\mathcal{O}) \\
		                                &= [(mP)+(nP)+(-(m+n)P)-3(\mathcal{O})] \\
							            &\phantom{}\qquad -[((m+n)P)+(-(m+n)P)-2\mathcal{O}] \\
							            &= (l_{mP,nP})-(v_{(m+n)P})
	\end{align*}
	So, $$f_{m+n,P}=f_{m,P}\cdot f_{n,P}\cdot\frac{l_{mP,nP}}{v_{(m+n)P}}$$

\end{frame}

\begin{frame}{Weil Pairing - Definition}

	For $P,Q\in E(\mathbb{F}_{q^k})[r]$, \vocab{Weil pairing} is defined as
	$$e(P,Q)=\frac{f_P(D_Q)}{f_Q(D_P)}$$
	where $U,V\in E(\mathbb{F}_{q^k})$ such that $\{P+U,U\}\cap\{Q+V,V\}=\varnothing$ and $f_P,f_Q$ are functions on the elliptic curve with divisors $(f_P)=r(P+U)-r(U)$ and $(f_Q)=r(Q+V)-r(V)$, and $D_P=(P+U)-(U)$ and $D_Q=(Q+V)-(V)$.
	
	Hence,
	$$e(P,Q)=\frac{f_P((Q+V)-(V))}{f_Q((P+U)-(U))}=\frac{f_P(Q+V)}{f_P(V)}\cdot\frac{f_Q(U)}{f_Q(P+U)}$$

\end{frame}

\begin{frame}{Weil Pairing - Well-Defined and Bilinearity}

	The reason that $U$ and $V$ are introduced is that $(f)=r(P)-r(\mathcal{O})$ and $D=(Q)-(\mathcal{O})$ do not have disjoint supports. Therefore, alternatives, which can be shown to make no difference {\color{blue}\cite{ref:silverman}(Silverman)}, are used.

	For bilinearity, let $Q_3=Q_1+Q_2$, and then
	$$e(P,Q_1+Q_2)={\color{blue}\frac{f_P(Q_1+Q_2+V_3)}{f_P(V_3)}}\cdot{\color{red}\frac{f_{Q_3}(U)}{f_{Q_3}(P+U)}}$$
	and
	$$e(P,Q_1)\cdot e(P,Q_2)={\color{blue}\frac{f_P(Q_1+V_1)}{f_P(V_1)}}\cdot{\color{red}\frac{f_{Q_1}(U)}{f_{Q_1}(P+U)}}\cdot{\color{blue}\frac{f_P(Q_2+V_2)}{f_P(V_2)}}\cdot{\color{red}\frac{f_{Q_2}(U)}{f_{Q_2}(P+U)}}$$

\end{frame}

\begin{frame}{Weil Pairing - Bilinearity}

	Letting $V_2=Q_1+V_1$ and $V_1=V_3$, we see
	$${\color{blue}\frac{f_P(Q_1+Q_2+V_1)}{f_P(V_1)}}={\color{blue}\frac{f_P(Q_1+V_1)}{f_P(V_1)}}\cdot{\color{blue}\frac{f_P(Q_2+(Q_1+V_1))}{f_P(Q_1+V_1)}}$$
	For the other parts, we need to show that ${\color{red}f(P+U)=f(U)}$ where ${\color{red}f=\frac{f_{Q_3}}{f_{Q_1}f_{Q_2}}}$. We get
	$$(f)=(f_{Q_3})-(f_{Q_1})-(f_{Q_2})=r[(Q_1+Q_2)-(Q_1)-(Q_2)+(\mathcal{O})]$$
	This implies $f=g^r$ where
	$$g=\frac{l_{Q_1,Q_2}}{v_{Q_3}}$$
	and the rest is analyzing these simple line functions.


\end{frame}

\begin{frame}{Weil Pairing - Additional Notes}

	Note that the pairing results are in $\mu_r\subset \mathbb{F}_{q^k}^\times$ because
	$$e(P,Q)^r=e(rP,Q)=e(\mathcal{O},Q)=\frac{f_\mathcal{O}(Q+V)}{f_\mathcal{O}(V)}\cdot\frac{f_Q(U)}{f_Q(\mathcal{O}+U)}=1$$
	and $(f_\mathcal{O})=r(\mathcal{O}+U)-r(U)=\varnothing$, which means $f_\mathcal{O}$ is constant.

	\vspace{1cm}

	Weil pairing is an example of a pairing used in cryptographic protocols, with a few others such as Tate pairing and Ate pairing, whose definitions are not much different.

\end{frame}

\section{Applications}

\begin{frame}{Applications}

	Remember that the pairing $e$ for cyclic groups $G_1=\langle P\rangle$, $G_2=\langle Q\rangle$ satisfy

	$$e(aP,bQ)=e(P,Q)^{ab}$$

	We will use this a lot in the applications.

\end{frame}

\subsection{BLS Signatures}

\begin{frame}{BLS Signatures - The Problem and the Setting}

	\textbf{The Problem:} {\color{red} A} wants to convince the public that they sign a message $m$. The message can be a document, or a financial transaction such as "{\color{red} A} gives \$100 to {\color{green} B}"

	\vspace{1cm}

	\textbf{The Setting:} Assume we have a pairing $e:G_1\times G_2\rightarrow G_T$ such that all groups are of order $p$, a huge prime number. Also assume we have a function $H:\{\text{message string}\}\rightarrow G_1$, available to everyone. This is just a convenience function for converting a string into a $G_1$ group element.
	
\end{frame}

\begin{frame}{BLS Signatures - Key Generation}
	
	\textbf{Key Generation:} {\color{red} A} chooses $0\leq x\leq p-1$ and constructs (secret key, public key) pair $(x, xG)$ where $G$ is a generator of $G_2$. {\color{red} A} shares their public key $xG$ publicly. Everyone knows the public key $xG$, but only {\color{red} A} knows $x$. And since Discrete Log Problem is \underline{hard}, nobody can obtain $x$ from $xG$.

	$$\boxed{x\text{ (secret key)}\rightarrow xG\text{ (public key)}}$$
	
	Let $m$ be the message that {\color{red} A} wants to sign.
	
\end{frame}

\begin{frame}{BLS Signatures - Sign and Verify}

	\textbf{Sign:} {\color{red} A} computes the signature $\sigma=x\,H(m)\in G_1$. Note that the signature can be created only by {\color{red} A}.

	\textbf{Verify:} Given a signature $\sigma$, and a message $m$, \underline{anyone} can verify that the message $m$ is signed by {\color{red} A}. All they have to do is to check the following equality: $$\boxed{e(\sigma,G)\overset{?}{=}e(H(m),xG)}$$
	
\end{frame}

\subsection{KZG Polynomial Commitment}

\begin{frame}{KZG Polynomial Commitment - Setting}

	\textbf{The Problem:} {\color{red} A} has a polynomial $P(x)=\sum_{i=0}^n a_ix^i$ and wants to prove that $P(u)=v$ to {\color{green} B} without revealing the polynomial, where $u$ is a value provided by {\color{green} B}.

	\vspace{1cm}

	This protocol is a member of an important family of protocols, namely \textbf{Zero Knowledge Protocols}. In ZK protocols, the prover convinces the other party only that a condition is satisfied, and nothing more.
	
\end{frame}

\begin{frame}{KZG Polynomial Commitment - Setup}

	\textbf{Setup:} Degree of the polynomial, $n$, is known to both {\color{red} A} and {\color{green} B}. We have a pairing $e:G_1\times G_2\rightarrow G_T$. 
	
	$$\boxed{\tau\rightarrow (G, \tau G,\tau^2 G,\cdots,\tau^n G)\text{ and }(H, \tau H)}$$

	A scalar $0\leq \tau\leq |G_1|-1$ is generated randomly, $(G, \tau G,\tau^2 G,\cdots,\tau^n G)$ and $(H, \tau H)$ are computed, and then $\tau$ is forgotten. ($\tau$ is created either jointly by {\color{red} A} and {\color{green} B} in such a way that they can not recover it by themselves, or it is created by a trusted third party {\color{blue} C})
	
\end{frame}

\begin{frame}{KZG Polynomial Commitment - Committing to the Polynomial}

	\textbf{Commit:} {\color{red} A} calculates the commitment $$\textbf{Commitment}=P(\tau)G=\sum_{i=0}^n a_i(\tau^iG)$$ and shares it with {\color{green} B}. This plays the role of a locked box, {\color{red} A} commits to the polynomial and can not change it later.
	
\end{frame}

\begin{frame}{KZG Polynomial Commitment - Proving}

	\textbf{Prove:} After receiving $u$ from {\color{green} B}, {\color{red} A} calculates $v=P(u)$. If {\color{red} A} is honest and following the protocol, there exists a polynomial $Q(x)=\sum_{i=1}^n b_ix^i$ such that $P(x)=Q(x)(x-u)+v$. The proof is then $$\textbf{Proof}=Q(\tau)G=\sum_{i=0}^n b_i(\tau^iG)$$
	
\end{frame}

\begin{frame}{KZG Polynomial Commitment - Verifying}

	\textbf{Verify:} After receiving the result $v$, and the proof $Q(\tau)G$ (and also the commitment $P(\tau)G$), {\color{green} B} performs the following pairing calculation to verify the proof:
	$$\boxed{e(P(\tau)G,H)\overset{?}{=}e(Q(\tau)G,\tau H-uH)\cdot e(G,H)^{v}}$$

	This process ensures that
	$$P(\tau)=Q(\tau)(\tau-u)+v$$
	Here, $\tau$ plays the role of a random value that neither {\color{red} A} nor {\color{green} B} has control of, but they can perform operations involving it. We rely on the fact that if an equation holds for a random value, then it probabilistically holds in general. So, {\color{green} B} is convinced that
	$$P(x)=Q(x)(x-u)+v$$
	which is equivalent to $P(u)=v$.
	
\end{frame}

\begin{frame}[plain]

	\begin{center}
		\LARGE{Thank you}
	\end{center}

\end{frame}

\appendix

\begin{frame}[allowframebreaks]{References}
	\begin{thebibliography}{9}
        \bibitem{ref:silverman} J. H. Silverman. The Arithmetic of Elliptic Curves (2nd Edition). Number 106 in Graduate Texts in Mathematics. Springer-Verlag, 2009.
		\bibitem{ref:galbraith} S. D. Galbraith. Mathematics of Public Key Cryptography. Cambridge University Press, March 2012
    \end{thebibliography}
\end{frame}

\end{document}
