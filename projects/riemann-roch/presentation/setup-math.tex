% Mathematical Writing
\DeclarePairedDelimiter{\abs}{\vert}{\vert}
\DeclarePairedDelimiter{\norm}{\Vert}{\Vert}
\DeclarePairedDelimiter{\ceil}{\lceil}{\rceil}
\DeclarePairedDelimiter{\floor}{\lfloor}{\rfloor}

\newcommand*{\inv}[1]{\ensuremath{#1^{-1}}}
\newcommand*{\positive}[1]{\ensuremath{\left[#1\right]^{+}}}

\newcommand*{\diff}{\ensuremath{\mathrm{d}}}
\newcommand*{\imag}{\ensuremath{\mathrm{j}}}
\newcommand*{\e}{\ensuremath{\mathrm{e}}}

\DeclareMathOperator*{\argmax}{arg\,max}
\DeclareMathOperator*{\argmin}{arg\,min}

%% change these to \mathbb, if you do not want to use the dsfont package
\newcommand*{\reals}{\ensuremath{\mathds{R}}} 
\newcommand*{\complexes}{\ensuremath{\mathds{C}}}
\newcommand*{\naturals}{\ensuremath{\mathds{N}}}
%%

\newcommand*{\expect}[2][]{\ensuremath{\mathbb{E}_{#1}\left[#2\right]}}

\newcommand*{\unif}{\ensuremath{\mathcal{U}}}
\newcommand*{\normaldist}{\ensuremath{\mathcal{N}}}


\newcommand{\mbx}[1]{\makebox[5cm]{#1\hfill}}
\newcommand{\poly}[2]{#1_0+#1_1x+\cdots+#1_#2x^#2}
\newcommand{\polyh}[2]{#1_0(y,z)+#1_1(y,z)x+\cdots+#1_#2(y,z)x^#2}

\newcommand{\curveC}{\mathcal C}
\newcommand{\curveD}{\mathcal D}
\newcommand{\resPQ}{\mathcal R_{P,Q}}
\newcommand{\CNZ}{\CC^{n+1}-\{\vec{0}\}}
\newcommand{\projective}{\mathbb P}
\DeclareMathOperator{\Div}{Div}
\DeclareMathOperator{\Res}{Res}
\DeclareMathOperator{\ord}{ord}

\newcommand{\CC}{\mathbb C}

\newcommand{\vocab}[1]{\textbf{\color{blue}\sffamily #1}}

% THEOREMS
\theoremstyle{plain}% default
% \newtheorem{theorem}{Theorem}%[section]
% \newtheorem{lemma}{Lemma}
% \newtheorem{proposition}{Proposition}
% \newtheorem{corollary}{Corollary}
\newtheorem{factx}[theorem]{Fact}
\newtheorem{observation}[theorem]{Observation}
\newtheorem{remark}[theorem]{Remark}
% \newtheorem{example}{Example}

\setbeamertemplate{theorems}[numbered]
