% !TeX program = lualatex
\documentclass[
	american,
	sections numbered,
	usenames,
	xcolor=dvipsnames,
	aspectratio=169,
]{beamer}

\mode<presentation>

\usepackage{babel}
\usepackage[babel]{microtype}
\usepackage[babel]{csquotes}
\usepackage[american]{isodate}

\usepackage[T1]{fontenc}
\usepackage{FiraMono}

\usetheme[progressbar=frametitle]{metropolis}

%%% GRAPHICS %%%
\usepackage{graphicx}
\usepackage{pgfplots}
\usepackage{tikz}
\usetikzlibrary{arrows.meta}

%%% MATH & SCIENCE %%%
\usepackage{amsmath}
\usepackage{amssymb}
\usepackage{amsfonts}
\usepackage{amsthm}
\usepackage{siunitx}
\usepackage{bm}
\usepackage{dsfont}
\usepackage{mathtools}

%%% FLOATS %%%
\usepackage{booktabs}
\usepackage{tabularx}

%\usepackage{biblatex}
%\bibliography{literature.bib}


% DESIGN COLORS
\definecolor{accent}{HTML}{7EBDC2} % accent color
\definecolor{bgcolor}{HTML}{FCFCFF} % background color
\definecolor{bgcolorAlt}{HTML}{ECF1FC} % alternative background color
\definecolor{fgcolor}{HTML}{222244} % foreground/text color

%
\setbeamercolor{normal text}{%
	fg=fgcolor,
	bg=bgcolor,
}
\setbeamercolor{alerted text}{%
	fg=accent,
}
\setbeamercolor{palette primary}{%
	use=normal text,
	fg=normal text.fg,
	bg=bgcolorAlt,%normal text.bg
}

\setbeamercolor{block title}{
	bg=bgcolorAlt,
}
\setbeamercolor{block body}{
	bg=bgcolorAlt,
}
\setbeamercolor{block title alerted}{%
	use={palette primary, alerted text},
	fg=palette primary.bg,
	bg=alerted text.fg
}
\setbeamercolor{block title example}{%
	use={block title, alerted text},
	bg=block title.bg,
	fg=alerted text.fg
}
%

\pgfplotsset{legend style={fill=bgcolor,draw=fgcolor}}

% PLOT COLORS
%% Paul Tol High Contrast
\definecolor{plot0}{HTML}{004488}
\definecolor{plot1}{HTML}{DDAA33}
\definecolor{plot2}{HTML}{BB5566}
\definecolor{plot3}{HTML}{000000}
\definecolor{plot4}{HTML}{AAAAAA}

%% Paul Tol Vibrant
%\definecolor{plot0}{HTML}{EE7733}
%\definecolor{plot1}{HTML}{0077BB}
%\definecolor{plot2}{HTML}{33BBEE}
%\definecolor{plot3}{HTML}{EE3377}
%\definecolor{plot4}{HTML}{CC3311}
%\definecolor{plot5}{HTML}{009988}
%\definecolor{plot6}{HTML}{BBBBBB}

\pgfplotscreateplotcyclelist{lineplot cycle}{ %
	{plot0, mark=*, thick, mark options=solid},
	{plot1, mark=triangle*, thick, mark options=solid},
	{plot2, mark=square*, thick, mark options=solid},
	{plot3, mark=diamond*, thick, mark options=solid},
	{plot4, mark=pentagon*, thick, mark options=solid},
}

% \AtBeginEnvironment{thm}{%
%   % \setbeamercolor{block title}{use=example text,fg=white,bg=example text.fg!75!black}
%   \setbeamercolor{block body}{parent=normal text,use=block title example,bg=blue!75!black!10!}
% }


%\renewcommand*{\bibfont}{\scriptsize}
\setbeamerfont{block body reference}{size=\scriptsize}
\setbeamerfont{block title reference}{size=\scriptsize}

\setbeamerfont{description item}{series=\mdseries}
\setbeamerfont{alerted text}{series=\bfseries\boldmath}


\setbeamertemplate{title page}{
\begin{minipage}[b][\textheight]{\textwidth}
	\ifx\inserttitlegraphic\@empty\else\usebeamertemplate*{title graphic}\fi
	\vfill%
	\ifx\inserttitle\@empty\else\usebeamertemplate*{title}\fi
	\ifx\insertsubtitle\@empty\else\usebeamertemplate*{subtitle}\fi
	\usebeamertemplate*{title separator}

	\ifx\beamer@shortauthor\@empty\else\usebeamertemplate*{author}\fi
	\ifx\insertdate\@empty\else\usebeamertemplate*{date}\fi
	\ifx\insertinstitute\@empty\else\usebeamertemplate*{institute}\fi
	\vfil
	\vspace*{1mm}
\end{minipage}
}
\newcommand*{\seprule}{{\par\color{bgcolorAlt!90!fgcolor}\hrulefill\par\vspace*{1ex plus 0pt minus .5ex}}}


% Mathematical Writing
\DeclarePairedDelimiter{\abs}{\vert}{\vert}
\DeclarePairedDelimiter{\norm}{\Vert}{\Vert}
\DeclarePairedDelimiter{\ceil}{\lceil}{\rceil}
\DeclarePairedDelimiter{\floor}{\lfloor}{\rfloor}

\newcommand*{\inv}[1]{\ensuremath{#1^{-1}}}
\newcommand*{\positive}[1]{\ensuremath{\left[#1\right]^{+}}}

\newcommand*{\diff}{\ensuremath{\mathrm{d}}}
\newcommand*{\imag}{\ensuremath{\mathrm{j}}}
\newcommand*{\e}{\ensuremath{\mathrm{e}}}

\DeclareMathOperator*{\argmax}{arg\,max}
\DeclareMathOperator*{\argmin}{arg\,min}

%% change these to \mathbb, if you do not want to use the dsfont package
\newcommand*{\reals}{\ensuremath{\mathds{R}}} 
\newcommand*{\complexes}{\ensuremath{\mathds{C}}}
\newcommand*{\naturals}{\ensuremath{\mathds{N}}}
%%

\newcommand*{\expect}[2][]{\ensuremath{\mathbb{E}_{#1}\left[#2\right]}}

\newcommand*{\unif}{\ensuremath{\mathcal{U}}}
\newcommand*{\normaldist}{\ensuremath{\mathcal{N}}}


\newcommand{\mbx}[1]{\makebox[5cm]{#1\hfill}}
\newcommand{\poly}[2]{#1_0+#1_1x+\cdots+#1_#2x^#2}
\newcommand{\polyh}[2]{#1_0(y,z)+#1_1(y,z)x+\cdots+#1_#2(y,z)x^#2}

\newcommand{\curveC}{\mathcal C}
\newcommand{\curveD}{\mathcal D}
\newcommand{\resPQ}{\mathcal R_{P,Q}}
\newcommand{\CNZ}{\CC^{n+1}-\{\vec{0}\}}
\newcommand{\projective}{\mathbb P}
\DeclareMathOperator{\Div}{Div}
\DeclareMathOperator{\Res}{Res}
\DeclareMathOperator{\ord}{ord}

\newcommand{\CC}{\mathbb C}

\newcommand{\vocab}[1]{\textbf{\color{blue}\sffamily #1}}

\newcommand{\hgline}[2]{
\pgfmathsetmacro{\thetaone}{#1}
\pgfmathsetmacro{\thetatwo}{#2}
\pgfmathsetmacro{\theta}{(\thetaone+\thetatwo)/2}
\pgfmathsetmacro{\phi}{abs(\thetaone-\thetatwo)/2}
\pgfmathsetmacro{\close}{less(abs(\phi-90),0.0001)}
\ifdim \close pt = 1pt
    \draw[blue] (\theta+180:1) -- (\theta:1);
\else
    \pgfmathsetmacro{\R}{tan(\phi)}
    \pgfmathsetmacro{\distance}{sqrt(1+\R^2)}
    \draw[blue] (\theta:\distance) circle (\R);
\fi
}

\newcommand\anglex{10}

\renewcommand{\H}{\mathbb{H}}
\DeclareMathOperator{\PSL}{PSL}

% THEOREMS
\theoremstyle{plain}% default
% \newtheorem{theorem}{Theorem}%[section]
% \newtheorem{lemma}{Lemma}
% \newtheorem{proposition}{Proposition}
% \newtheorem{corollary}{Corollary}
\newtheorem{factx}[theorem]{Fact}
\newtheorem{observation}[theorem]{Observation}
\newtheorem{remark}[theorem]{Remark}
% \newtheorem{example}{Example}

\setbeamertemplate{theorems}[numbered]

\pgfplotsset{compat=newest}
\pgfplotsset{%
	betterplot/.style={
		width=.93\linewidth,
		height=.5\textheight,
		xlabel near ticks,
		ylabel near ticks,
		cycle list name=lineplot cycle,
		mark options=solid,
		xmajorgrids=true,
		xminorgrids=true,
		ymajorgrids=true,
%		major grid style={dotted},
		grid style={line width=.1pt, draw=gray!20},
		major grid style={line width=.25pt,draw=gray!30},
		legend cell align=left,
		legend style = {
			/tikz/every even column/.append style={column sep=0.33cm}
		},
	},
}


\title{Geometrically Finite Fuchsian Groups and Poincare Theorem}
\author{Hakan Karakuş}
\date{MATH58F}

% \titlegraphic{\includegraphics[width=0.3\textheight]{figures/logo.png}}

\begin{document}
\begin{frame}[plain]
	\titlepage
\end{frame}

% \begin{frame}{Table of contents}
	\setbeamertemplate{section in toc}[sections numbered]
	\tableofcontents%[hideallsubsections]
\end{frame}

\section{Introduction}

\begin{frame}{Introduction}

	Curves have been interesting for a long time. Intuitively, a curve may be thought of as the trace left by a moving point. The first definition of a curve in the literature of mathematics appeared in Euclid's Elements.

	\begin{exampleblock}{}
	  	{\large ``The [curved] line is the first species of quantity, which has only one dimension, namely length, without any width nor depth, and is nothing else than the flow or run of the point which will leave from its imaginary moving some vestige in length, exempt of any width.''}
	  	\vskip5mm
	  	\hspace*\fill{\small--- Euclid's Elements}
	\end{exampleblock}

\end{frame}

\begin{frame}{Introduction}
	
	In modern mathematics, curves have various definitions depending on the settings they are in. In this setting, the main actor is the algebraic curves which are the zero set of polynomials defined over the field of complex numbers.

	An interesting question about curves is to find functions on them with prescribed zeroes and poles. The \textbf{Riemann-Roch theorem} finds the dimension of the space of such meromorphic functions. This theorem is a vital tool in the fields of complex analysis and algebraic geometry. It relates the complex analysis of a connected compact Riemann surface with the surface's purely topological property of genus, in a way that can be carried over into purely algebraic settings.

\end{frame}

\begin{frame}{Introduction - Riemann-Roch Theorem}

	\begin{theorem}[Riemann-Roch]
		Let $D$ be a divisor on a non-singular projective curve $\curveC$ in $\projective_2$ with genus $g$, $\kappa$ be a canonical divisor on $\curveC$.
		$$l(D)-l(\kappa-D)=\deg(D)-g+1$$
	\end{theorem}

\end{frame}

\begin{frame}{Introduction - Theorem Consequences and Other Proofs}

	The Riemann-Roch theorem has many very useful consequences including an easy proof of the law of associativity for the additive group structure on a non-singular cubic curve and a proof that every meromorphic function on a non-singular projective curve is rational.

	There are different ways to prove the Riemann–Roch Theorem. One way is to take an analytic approach and study holomorphic and meromorphic functions with Serre duality, which can be found in \cite{ref:miranda}. Another, more modern approach includes the concept of schemes and sheaf cohomology, and an example to this can be found in \cite{ref:hartshorne}. I have utilized resources \cite{ref:kirwan}, \cite{ref:fulton}, \cite{ref:keith}, \cite{ref:hampus}, \cite{ref:terrytao} to study the theorem. This paper uses a more elementary machinery to approach the Riemann-Roch theorem.

\end{frame}

\section{Preliminaries}

\begin{frame}{Preliminaries}

	In order to understand the statement and the proof of the Riemann–Roch theorem, one must first state several definitions and results.

\end{frame}

\subsection{Complex Curves}

\begin{frame}{Complex Algebraic Curves in $\CC^2$}

	A \vocab{complex algebraic curve} in $\CC^2$ is
	$$\curveC=\{(x,y)\in\CC^2:P(x,y)=0\}$$
	where $P(x,y)\in\CC[x,y]$ is a polynomial with no repeated factors.
	$$d=\max\{u+v:c_{u,v}\neq 0\}$$ where $\displaystyle P(x,y)=\sum_{u,v}c_{u,v}\,x^u\,y^v$ is called the \vocab{degree} of the curve.

\end{frame}

\begin{frame}{Hilbert's Nullstellensatz}
	
	The condition of no repeated factors is needed because of the following theorem.

	\begin{factx}[Hilbert's Nullstellensatz]
		For $P(x,y),Q(x,y)\in\CC[x,y]$, $\curveC_P=\curveC_Q$ if and only if $P$ and $Q$ have the same irreducible factors, possibly occuring with different multiplicities.
	\end{factx}

\end{frame}

\begin{frame}{Singularity, Multiplicity and Tangent Lines}
	
	$(a,b)\in\curveC$ is a \vocab{singularity} of $\curveC$ if $$\frac{\partial P}{\partial x}(a,b)=\frac{\partial P}{\partial y}(a,b)=0$$
	The \vocab{multiplicity} of $\curveC$ at $(a,b)\in\curveC$ is the smallest positive integer $m$ such that $$\frac{\partial^mP}{\partial x^i\partial y^j}(a,b)\neq 0$$ for some $i\geq 0$, $j\geq 0$ such that $i+j=m$. The Taylor polynomial $$\sum_{i+j=m}\frac{\partial^mP}{\partial x^i\partial y^j}(a,b)\frac{(x-a)^i(y-b)^j}{i!j!}$$ is then homogeneous of degree $m$ and its linear factors (homogeneous polynomials factor as a product of linear polynomials) are the tangent lines to $\curveC$ at $(a,b)$. If the factors of this polynomial are all different lines, than this singularity is \vocab{ordinary}.

\end{frame}

\begin{frame}{Example}
	
	The cubic curve (degree 3) defined by the polynomial $P(x,y)=x^3+x^2-y^2$ has a double point (a singularity of multiplicity 2) at the point $(0,0)$ which is ordinary.
	Although they are fundamentally different, the real algebraic curve counterpart has the following shape.
	\begin{center}
		\begin{tikzpicture}[scale=0.8]
			\draw[->] (-2.2, 0) -- (2.2, 0) node[right] {$x$};
			\draw[->] (0, -2.2) -- (0, 2.2) node[above] {$y$};
			\draw[scale=2.0, domain=-1:1, smooth, variable=\x, blue] plot (\x,{sqrt((\x*\x*(\x+1))});
			\draw[scale=2.0, domain=-1:1, smooth, variable=\x, blue] plot (\x,{-sqrt((\x*\x*(\x+1))});
		\end{tikzpicture}
	\end{center}

\end{frame}

\begin{frame}{Complex Projective Space}
	
	\vocab{Complex projective space} of dimension $n$, denoted $\projective_n$, is the set of complex one dimensional subspaces of the complex vector space $\CC^{n+1}$. $\projective_n$ can be identified with the set of equivalence classes for the equivalence relation $\sim$ on $\CNZ$ such that $a\sim b$ if there is some $\lambda\in\CC-\{0\}$ such that $a=\lambda b$. That is, every vector of $\CNZ$ represents an element $x$ of $\projective_n$
	$$\projective_n=\{[x_0,\cdots,x_n]:(x_0,\cdots,x_n)\in\CNZ\}$$
	such that
	$$[\lambda x_0,\lambda x_1,\cdots,\lambda x_n]\sim[x_0,x_1,\cdots,x_n]$$

\end{frame}

\begin{frame}{Complex Projective Space}
	
	\begin{remark}
		$\projective_n$ is $\CC^n$ with a copy of $\projective_{n-1}$ at infinity.
	\end{remark}

	\seprule

	\begin{itemize}
		\item Adding a point at infinity to the complex plane results in a space that is topologically a sphere. Hence the complex projective line, $\projective_1$, is also known as the Riemann sphere.
	\end{itemize}

\end{frame}

\begin{frame}{Complex Projective Curves in $\projective_2$}
	
	A \vocab{complex projective curve} in $\projective_2$ is
	$$\curveC=\{[x,y,z]\in \projective_2:P(x,y,z)=0\}$$
	where $P(x,y,z)\in\CC[x,y,z]$ is a non-constant homogeneous polynomial with no repeated factors. $d=\deg P$ is called the \vocab{degree} of the curve.

\end{frame}

\begin{frame}{Singularity, Multiplicity and Tangent Lines}
	
	$[a,b,c]\in\projective_2$ is a \vocab{singularity} of $\curveC$ if $$\frac{\partial P}{\partial x}(a,b,c)=\frac{\partial P}{\partial y}(a,b,c)=\frac{\partial P}{\partial z}(a,b,c)=0$$
	The \vocab{multiplicity} of $\curveC$ at $[a,b,c]\in\curveC$ is the smallest positive integer $m$ such that $$\frac{\partial^mP}{\partial x^i\partial y^j\partial z^k}(a,b,c)\neq 0$$ for some $i\geq 0$, $j\geq 0$, $k\geq 0$ such that $i+j+k=m$.
	Tangent line to $\curveC$ at a non-singular point $[a,b,c]$ is the line $$\frac{\partial P}{\partial x}(a,b,c)x+\frac{\partial P}{\partial y}(a,b,c)y+\frac{\partial P}{\partial z}(a,b,c)z=0$$

	\seprule

	\begin{itemize}
		\item The curve defined by $P(x,y,z)=y^2z-x^3$ has a singular point at $[0,0,1]$
	\end{itemize}

\end{frame}

\begin{frame}{Affine and Projective Curves}
	
	Complex affine curves and complex projective curves are different but closely related by the following lemma.

	\begin{lemma}\label<1>{lemma:affproj}
		Let $[a,b,c]$ be a point of the projective curve $$\tilde{\curveC}=\{[x,y,z]\in \projective_2:P(x,y,z)=0\}$$
		If $c\neq 0$, then $[a,b,c]$ is a non-singular point of $\tilde{\curveC}$ if and only if $\left(\frac{a}{c},\frac{b}{c}\right)$ is a non-singular point of the affine curve $$\curveC=\{(x,y)\in\CC^2:P(x,y,1)=0\}$$
	\end{lemma}

\end{frame}

\begin{frame}{Proof of Lemma \ref{lemma:affproj}}
	
	The point $\left(\frac{a}{c},\frac{b}{c}\right)$ is a singular point of $\curveC$ if and only if $$P\left(\frac{a}{c},\frac{b}{c},1\right)=0=\frac{\partial P}{\partial x}\left(\frac{a}{c},\frac{b}{c},1\right)=\frac{\partial P}{\partial y}\left(\frac{a}{c},\frac{b}{c},1\right)$$
	which by $P$ being homogeneous is equivalent to
	$$P(a,b,c)=0=\frac{\partial P}{\partial x}(a,b,c)=\frac{\partial P}{\partial y}(a,b,c)$$
	By the Euler relation given below
	$$x\frac{\partial P}{\partial x}(x,y,z)+y\frac{\partial P}{\partial y}(x,y,z)+z\frac{\partial P}{\partial z}(x,y,z)=dP(x,y,z)$$
	which can be obtained by differentiating
	$$P(\lambda x,\lambda y,\lambda z)=\lambda^dP(x,y,z)$$
	with respect to $\lambda$ and then setting $\lambda=1$, this happens when $\frac{\partial P}{\partial z}(a,b,c)=0$ also holds, i.e. $[a,b,c]$ is a singular point of $\tilde{\curveC}$.

\end{frame}

\subsection{Bezout Theorem}

\begin{frame}{Bezout Theorem}
	
	\begin{theorem}[Bezout]\label<1>{thm:bezout}
		$\curveC$ and $\curveD$ are complex projective curves of degrees $n$ and $m$ in $\projective_2$ which have no common component. $$\sum_{p\in \curveC\cap\curveD}I_p(\curveC,\curveD)=nm$$
	\end{theorem}

\end{frame}

\begin{frame}{Resultant}
	
	Let $$P(x)=\poly{a}{n},$$ $$Q(x)=\poly{b}{m},$$ be polynomials of degrees $n$ and $m$ in $\CC[x]$. The \vocab{resultant} $\resPQ$ is the determinant of the $n+m$ by $n+m$ matrix.

	For polynomials $$P(x,y,z)=\polyh{a}{n},$$ $$Q(x,y,z)=\polyh{b}{m},$$ the resultant $\resPQ(y,z)$ is defined similarly.

\end{frame}

\begin{frame}{Resultant Matrix}
	
	$$
	\begin{pmatrix}
		a_0 & a_1 & \cdots & a_n & 0 & 0 & & & \cdots & 0 \\
		0 & a_0 & a_1 & \cdots & a_n & 0 & & & \cdots & 0 \\
		\vdots & & & & & & & & & \vdots \\
		0 & 0 & \cdots & 0 & a_0 & a_1 & & & \cdots & a_n \\
		b_0 & b_1 & & \cdots & & b_m & 0 & & \cdots & 0 \\
		0 & b_0 & b_1 & & \cdots & & b_m & 0 & \cdots & 0 \\
		\vdots & & & & & & & & & \vdots \\
		0 & \cdots & 0 & b_0 & b_1 & & & & \cdots & b_m
	\end{pmatrix}
	$$

\end{frame}

\begin{frame}{Resultant Properties}
	
	\begin{itemize}
		\item If $P(x,y,z)$ and $Q(x,y,z)$ are homogeneous polynomials of degrees $n$ and $m$, then $\resPQ(y,z)$ is either identically 0 or a homogeneous polynomial of degree $nm$.
		
		\item Polynomials $P(x)$ and $Q(x)$ have a non-constant common factor if and only if $\resPQ=0$.
		
		\item Homogeneous polynomials $P(x,y,z)$ and $Q(x,y,z)$ have a non-constant common factor if and only if $\resPQ(y,z)\equiv 0$.
	
		\item If $P(x)=\prod_{i=0}^{n}x-\lambda_i$ and $Q(x)=\prod_{j=0}^{m}x-\mu_j$ then $$\resPQ=\prod_{0\leq i\leq n,0\leq j\leq m}\mu_j-\lambda_i$$
		In particular, $\mathcal{R}_{P,QR}=\mathcal{R}_{P,Q}\mathcal{R}_{P,R}$
	\end{itemize}

\end{frame}

\begin{frame}{Weak Form of Bezout Theorem}
	
	\begin{theorem}\label<1>{thm:weak_bezout}
		Any two projective curves $\curveC$ and $\curveD$ in $\projective_2$ of degrees $n$ and $m$ intersect in at least one point and at most $nm$ points if they have no common component.
	\end{theorem}

\end{frame}

\begin{frame}{Proof of Weak Form of Bezout Theorem}
	
	Let $\curveC$ and $\curveD$ are defined by homogeneous polynomials $P(x,y,z)$ and $Q(x,y,z)$ of degrees $n$ and $m$. The resultant $\resPQ(y,z)$ is a homogeneous polynomial of degree $nm$, so the product of $nm$ linear factors of the form $bz-cy$. For any factor of the form $bz-cy$, the resultant of polynomials $P(x,b,c)$ and $Q(x,b,c)$ in $x$ is identically 0. Therefore, there exists $a\in\CC$ such that $P(a,b,c)=Q(a,b,c)=0$. Since there is at least one such pair $(b,c)$, there is a point $[a,b,c]\in\curveC\cap\curveD$.

\end{frame}

\begin{frame}{Proof of Weak Form of Bezout Theorem (continued)}
	
	Now suppose that $\curveC$ and $\curveD$ have more than $nm$ intersections. Choose any finite set $S$ of distinct points in $\curveC\cap\curveD$ with more than $nm$ elements. By applying a suitable projective transformation, assume that $[1,0,0]$ is not on $\curveC$ or on $\curveD$ or on a line passing through two distinct points in $S$. For any intersection point, there must be a linear factor because $b$ and $c$ can not simultaneously be zero since $[1,0,0]$ is not an intersection point. Also, any two such linear factors are different because $[1,0,0]$ is not on a line passing through two intersection points. Hence, for any intersection point, $\resPQ(y,z)$ has a different linear factor but $\resPQ(y,z)$ with more than $nm$ linear factors must be identically zero.

\end{frame}

\begin{frame}{Intersection Multiplicity}

	There is a unique intersection multiplicity $I_P(\curveC,\curveD)$ defined for all projective curves $\curveC$, $\curveD$ in $\projective_2$ satisfying the following properties.
	\begin{itemize}
		\item $I_P(\curveC,\curveD)=I_P(\curveD,\curveC)$
		\item $I_P(\curveC,\curveD)=\infty$ if $P$ lies on a common component and otherwise $I_P(\curveC,\curveD)$ is a non-negative integer
		\item $I_P(\curveC,\curveD)=0$ if and only if $P\not\in\curveC\cap\curveD$
		\item Two distinct lines meet with intersection multiplicity of 1 at their unique point of intersection.
		\item If $\curveC:=P(x,y,z)$, $\curveC_1:=P_1(x,y,z)$, and $\curveC_2:=P_2(x,y,z)$ where $P(x,y,z)=P_1(x,y,z)P_2(x,y,z)$, then $I_P(\curveC,\curveD)=I_P(\curveC_1,\curveD)+I_P(\curveC_2,\curveD)$
		\item If $\curveC:=P(x,y,z)$, $\curveD:=Q(x,y,z)$ have degrees $n$ and $m$, and $\mathcal{E}:=PR+Q$ where $R(x,y,z)$ is of degree $m-n$, then $I_P(\curveC,\curveD)=I_P(\curveC,\mathcal{E})$
	\end{itemize}

\end{frame}

\begin{frame}{Intersection Multiplicity}

	Moreover, if $\curveC$ and $\curveD$ have no common component and by suitable selection $[1,0,0]$ does not belong to $\curveC\cap\curveD$, any line through two distinct points of $\curveC\cap\curveD$, or any tangent to $\curveC$ or $\curveD$ at a point of $\curveC\cap\curveD$, then $I_P(\curveC,\curveD)$ of any $P=[a,b,c]\in\curveC\cap\curveD$ is $\nu_{bz-cy}(\resPQ(y,z))$

\end{frame}

\begin{frame}{Proof of Bezout Theorem}
	
	The resultant can be express as
	$$\resPQ(y,z)=\prod_{i=0}^k(b_iz-c_iy)^{e_i}$$
	where $e_1+e_2+\cdots+e_k=nm$. By the arguments in the proof of the weak form of Bezout Theorem, $I_{P_i}(\curveC,\curveD)=e_i$.

\end{frame}

\subsection{Degree-Genus Formula for Non-singular Curves}

\begin{frame}{Degree-Genus Formula}
	
	A non-singular complex projective curve in $\projective_2$ is topologically a sphere with $g$ handles. This number $g$ is called the \vocab{genus} of the curve.
	
	\begin{theorem}[degree-genus formula]\label<1>{thm:degree_genus}
		For a non-singular complex projective curve of degree $d$ in $\projective_2$ with genus $g$, $$g=\frac{(d-1)(d-2)}{2}$$
	\end{theorem}

\end{frame}

\begin{frame}{Intuitive Proof of Degree-Genus Formula}

	Consider a singular complex projective curve $\curveC$ which is union of $d$ projective lines in general position i.e. no point lies on more than two lines. A complex projective line $L$ is homeomorphic to the two dimensional unit sphere $\mathbb{S}^2$ by the stereographic projection. So, $\curveC$ is homeomorphic to a union of $d$ spheres meeting in $\frac{d(d-1)}{2}$ points. It is possible to perturb the curve by a small amount to get a non-singular curve. This perturbation turns the $\frac{d(d-1)}{2}$ intersection points into handles expect for $d-1$ points which are used to join the $d$ spheres. Hence, the number of handles is $\frac{d(d-1)}{2}-(d-1)$.

\end{frame}

\subsection{Riemann Surfaces}

\begin{frame}{Riemann Surfaces}
	
	A \vocab{Riemann surface} is a connected complex one dimensional manifold which is locally homeomorphic to $\CC$ i.e any point has an open neighborhood homeomorphic to an open subset of $\CC$. The feature of Riemann surfaces that interests this paper the most is that it makes sense to do complex analysis and define holomorphic and meromorphic functions on them. A \vocab{holomorophic function} on a Riemann surface $S$ is a holomorphic map from $S$ to $\CC$ and a \vocab{meromorphic function} on $S$ is a holomorphic map from $S$ to $\overline{\CC}=\CC\cup\{\infty\}$. Sphere, torus, and the curves that this paper is dealing with are the most famous examples.

\end{frame}

\begin{frame}{Differentials}
	
	A \vocab{meromorphic differential} on $\curveC$ is a symbol $df$ which satisfies the following relations
	\begin{align*}
		d(f+g) &= df+dg\\
		d(fg) &= fdg+gdf\\
		da &= 0
	\end{align*}
	for any meromorphic functions $f$, $g$ on $\curveC$ and $a\in\CC$.

\end{frame}

\subsection{Divisors}

\begin{frame}{Divisors}
	
	A \vocab{divisor} $D$ on $\curveC$ is the formal sum $$D=\sum_{P\in\curveC}n_PP$$ such that $n_P\in\mathbb{Z}$ for every $P\in\curveC$ and $n_P=0$ for all but finitely many $p\in\curveC$.
	
	The \vocab{degree} of $D$ is then $$\deg(D)=\sum_{P\in\curveC}n_P$$ The set of all divisors on $\curveC$ is an Abelian group, denoted $\Div(\curveC)$, and the degree defines a homomorpism from $\Div(\curveC)$ to $\mathbb{Z}$.

\end{frame}

\begin{frame}{Effective Divisors}
	
	If $n_P\geq 0$ for all $P\in\curveC$, then $D$ is called \vocab{effective} and denoted $D\geq 0$. $D\geq D'$ if $D-D'\geq 0$. This also means $\deg(D)\geq\deg(D')$ 

\end{frame}

\begin{frame}{Principal Divisors}
	
	The divisor of a meromorphic function on $\curveC$
	$$(f)=\sum_{P\in\curveC}\ord_P(f)P$$
	is called a \vocab{principal divisor} where $\ord_P(f)$ is the order of zero (or the negative of the order of pole) at $P$. Two divisors $D$ and $D'$ are said to be \vocab{equivalent}, denoted $D\sim D'$, if $D-D'$ is a principal divisor.
	
	A principal divisor on $\curveC$ has degree zero. That is, any meromorphic function defined on $\curveC$ has the same number of zeros and poles, counted with multiplicities. Therefore, equivalent divisors have the same degree.

\end{frame}

\begin{frame}{Properties of Principal Divisors}
	
	\begin{minipage}{0.50\textwidth}
		\begin{align*}
			\ord_P(fg) &= \ord_P(f)+\ord_P(g) \\
			\ord_P\left(\frac{f}{g}\right) &= \ord_P(f)-\ord_P(g) \\
			\ord_P(f+g) &\geq \min(\ord_P(f),\ord_P(g))
		\end{align*}
	\end{minipage}
	\begin{minipage}{0.10\textwidth}
		$\implies$
	\end{minipage}
	\begin{minipage}{0.30\textwidth}
		\begin{align*}
			(fg) &= (f)+(g) \\
			\left(\frac{f}{g}\right) &= (f)-(g) \\
			(f+g) &\geq \min((f),(g))
		\end{align*}
	\end{minipage}

\end{frame}

\begin{frame}{Vector Space of Functions}
	
	Let $D=\sum_{P\in\curveC}n_PP$ be a divisor on $\curveC$, then $\mathcal{L}(D)$ is the set of meromorphic functions $f$ on $\curveC$ satisfying $(f)+D\geq 0$ together with the zero function. That is, a meromorphic function $f$ on $\curveC$ belongs to $\mathcal{L}(D)$ if $f$ is holomorphic except at those $P\in\curveC$ for which $n_P>0$ and there the order of the pole is at most $n_P$, and also $f$ has a zero of order at least $-n_P$ at every $P\in\curveC$ with $n_P<0$. By the properties of principal divisors, $\mathcal{L}(D)$ is a complex vector space. Denote $l(D)=\dim(\mathcal{L}(D))$.

	\seprule

	\begin{itemize}
		\item Let $D=2P_1-3P_2$. If $f\in\mathcal{L}(D)$, then it has a pole of order at most 2 at $P_1$ and a zero of order at least 3 at $P_2$.
	\end{itemize}

\end{frame}

\begin{frame}{Properties of $\mathcal{L}(D)$}
	
	\begin{factx}\label<1>{corollary:ldzero}
		If $\deg(D)<0$, then $l(D)=0$ because if $f$ is a meromorphic function on $\curveC$ such that $(f)+D\geq 0$, then $\deg(D)=\deg((f)+D)\geq 0$. Also, $\mathcal{L}(0)$ consists of only constant functions, so $l(0)=1$.
	\end{factx}

	\begin{factx}\label<1>{fact:ld12}
		If $D\sim D'$, then $l(D)=l(D')$ because if $D'=D+(g)$ where $g$ is a meromorphic function on $\curveC$, then $f\mapsto fg$ defines a isomorphism from $\mathcal{L}(D)$ to $\mathcal{L}(D')$. Also, if $D_1\leq D_2$, then $\mathcal{L}(D_1)\subset\mathcal{L}(D_2)$.
	\end{factx}

\end{frame}

\begin{frame}{Divisors}
	
	\begin{lemma}\label<1>{lemma:ldp}
		$0\leq l(D+P)-l(D)\leq 1$ for any divisor $D$ and point $P$ on $\curveC$.
	\end{lemma}
	
	By Fact \ref{fact:ld12}, $\mathcal{L}(D)\subset\mathcal{L}(D+P)$, so $0\leq l(D+P)-l(D)$. Take a function $t$ such that $\ord_P(t)=1$ (it is called uniformizer). Consider
	$$\phi:\mathcal{L}(D+P)\rightarrow \CC,\,f\mapsto (t^{n_P+1}f)(P)$$
	that maps $f$ to the evaluation of $t^{n_P+1}f$ at $P$. Since $\ord_P(t^{n_P+1}f)=n_P+1+\ord_P(f)\geq 0$ for a $f\in\mathcal{L}(D+P)$, $P$ is not a pole and $\phi$ is well-defined. $\ker(\phi)$ consists of functions that has a zero at $P$ i.e. $\ord_P(t^{n_P+1}f)=n_P+1+\ord_P(f)\geq 1$. This precisely describes the functions in $\mathcal{L}(D)$. Therefore,
	$$\mathcal{L}(D+P)/\mathcal{L}(D)\cong\CC$$
	and $\dim(\mathcal{L}(D+P)/\mathcal{L}(D))=l(D+P)-l(D)\leq 1$.

\end{frame}

\begin{frame}{Canonical Divisors}
	
	If $\omega$ is a meromorphic differential on $\curveC$ which is not identically zero, then we can define the divisor $(\omega)$ of $\omega$ similarly. The divisor of a meromorphic differential is called a \vocab{canonical divisor} and is denoted by $\kappa$. Any two canonical divisors are equivalent and have the same degree of $2g-2$ by Proposition 6.31 from \cite{ref:kirwan}.

\end{frame}

\section{Theorem Statement and Proof}

\begin{frame}{Theorem Statement}
	
	Let $D$ be a divisor on a non-singular projective curve $\curveC$ in $\projective_2$ with genus $g$, $\kappa$ be a canonical divisor on $\curveC$, and $P$ be a point on $\curveC$.

\end{frame}

\begin{frame}{Lemmas for the Riemann-Roch Theorem}
	
	\begin{lemma}\label<1>{lemma:ld_ldp}
		$$0\leq l(D)-l(D-P)+l(\kappa-D+P)-l(\kappa-D)\leq 1$$
	\end{lemma}
	
	By Lemma \ref{lemma:ldp}, assume simultaneously $l(D)-l(D-P)=1$ and $l(\kappa-D+P)-l(\kappa-D)=1$. Then there exists $f\in\mathcal{L}(D)-\mathcal{L}(D-P)$ and $g\in\mathcal{L}(\kappa-D+P)-\mathcal{L}(\kappa-D)$. So, $(f)+D\geq 0$ and $(g)+\kappa-D+P\geq 0$. These inequalities are actually equalities at point $P$. Adding the inequalities having this in mind, one gets $(fg)+\kappa+P\geq 0$ with an equality at $P$. So, $fg\in\mathcal{L}(\kappa+P)-\mathcal{L}(\kappa)$. This is a contradiction since $\mathcal{L}(\kappa+P)-\mathcal{L}(\kappa)=\varnothing$ by Proposition 5.15 from \cite{ref:hampus}.

\end{frame}

\begin{frame}{Lemmas for the Riemann-Roch Theorem}
	
	\begin{lemma}[Riemann's Inequality]\label<1>{lemma:halfrr}
		$$l(D)-l(\kappa-D)\geq\deg(D)+1-g$$
	\end{lemma}
	
	Let $L(x,y,z)$ define a line in $\projective_2$ and let
	$$H=\sum_{P\in\curveC}I_P(\curveC,L)P$$
	be a divisor of degree $d$ by Bezout theorem (Theorem \ref{thm:bezout}). $\deg(\kappa-mH)=\deg(\kappa)-md$ and this is negative for large enough $m$. By Corollary \ref{corollary:ldzero}, $l(\kappa-mH)=0$. For a homogeneous polynomial $Q(x,y,z)$ of degree $m$,
	$$\frac{Q(x,y,z)}{L(x,y,z)^m}$$
	defines a meromorphic function $f$ that satisfies $(f)+mH\geq 0$ i.e. an element of $\mathcal{L}(mH)$.

\end{frame}

\begin{frame}{Lemmas for the Riemann-Roch Theorem}
	
	For any such $Q(x,y,z)$, a function $f$ can be obtained but $Q'(x,y,z)=Q(x,y,z)+P(x,y,z)R(x,y,z)$ gives the same function $f$ for any homogeneous polynomial $R(x,y,z)$ of degree $m-d$. Hence, if $\CC_t[x,y,z]$ defines the space of homogeneous polynomials of degree $t$ whose dimension is easily found to be $|\{x^iy^jz^k:i+j+k=t\}|={t+2 \choose 2}=\frac{(t+1)(t+2)}{2}$, then
	\begin{align*}
		l(mH) &\geq \dim(\CC_m[x,y,z]/P(x,y,z)\CC_{m-d}[x,y,z])\\
		&= \dim(\CC_m[x,y,z])-\dim(\CC_{m-d}[x,y,z])\\
		&= \frac{(m+1)(m+2)}{2}-\frac{(m-d+1)(m-d+2)}{2}\\
		&= md-g+1\\
		&= \deg(mH)-g+1
	\end{align*}
	by the degree-genus formula (Theorem \ref{thm:degree_genus}). Thus, the lemma holds for $D=mH$ when $m$ is large enough.

\end{frame}

\begin{frame}{Lemmas for the Riemann-Roch Theorem}
	
	Now, the aim is, for any divisor $D=\sum_{P\in\curveC}n_PP$, and $m_0$, to find a $m\geq m_0$ and not necessarily distinct points $P_1,P_2,\cdots,P_k\in\curveC$ such that 
	$$D+P_1+P_2+\cdots+P_k\sim mH$$
	By adding points, $n_P\geq 0$ can be assumed. For each of the finitely many $P$ such that $n_P>0$, take a line through $P$ that intersects $\curveC$ at points (possibly with multiplicities) $Q_{1,P},Q_{2,P},\cdots,Q_{d,P}$. Since the ratio of any two linear homogeneous polynomials defines a meromorphic function, for any $P$, $$Q_{1,P}+Q_{2,P}+\cdots+Q_{d,P}\sim H$$
	Taking $m=\deg(D)$,
	$$mH\sim \sum_{n_P>0}n_P\sum_{i=1}^dQ_{i,P}= D+P_1+P_2+\cdots+P_k$$
	for $k=md-m$.

\end{frame}

\begin{frame}{Lemmas for the Riemann-Roch Theorem}
	
	So, by a simple induction on Lemma \ref{lemma:ld_ldp} and the results proven so far
	\begin{align*}
		l(D)-l(\kappa-D) &\geq l(D+P_1+P_2+\cdots+P_k)-l(\kappa-D-P_1-P_2-\cdots-P_k)-k\\
		&= l(mH)-l(\kappa-mH)-k\\
		&\geq \deg(mH)-g+1-k\\
		&= \deg(D+P_1+P_2+\cdots+P_k)-g+1-k\\
		&= \deg(D)-g+1
	\end{align*}
	as $l(mH)\geq\deg(mH)-g+1$ and $l(\kappa-mH)=0$.

\end{frame}

\begin{frame}[standout,plain]

	Now comes the main theorem that turns this inequality into an equality.
	
\end{frame}

\begin{frame}{Riemann-Roch Theorem}
	
	\begin{theorem}[Riemann-Roch]\label<1>{thm:rr}
		$$l(D)-l(\kappa-D)=\deg(D)-g+1$$
	\end{theorem}

\end{frame}

\begin{frame}{Proof of Riemann-Roch Theorem}
	
	Apply Lemma \ref{lemma:halfrr} for $D$ and $\kappa-D$ respectively to obtain
	$$l(D)-l(\kappa-D)\geq\deg(D)-g+1$$
	and
	\begin{align*}
		l(\kappa-D)-l(D) &\geq\deg(\kappa-D)-g+1\\
		&= \deg(\kappa)-\deg(D)-g+1\\
		&= 2g-2-\deg(D)-g+1\\
		&= -\deg(D)+g-1
	\end{align*}
	since $\deg(\kappa)=2g-2$.

\end{frame}

\section{Applications}

\begin{frame}{Applications - Degree versus Dimension of Vector Space}
	
	The relation between $\deg(D)$ and $l(D)$ for a curve of genus $g$ can be partially determined by the Riemann-Roch theorem (Theorem \ref{thm:rr}). $l(D)$ of a divisor of negative degree is 0 by Corollary \ref{corollary:ldzero}. For divisors $D$ with $\deg(D)>2g-2$, one gets $l(D)=\deg(D)-g+1$ since $l(\kappa-D)=0$ again by Corollary \ref{corollary:ldzero}. For each additional point added to $D$, $l(D)$ either stays the same or increases by 1 by Lemma \ref{lemma:ldp}. So, the graph of $\deg(D)$ versus $l(D)$ in $\mathbb{Z}\times\mathbb{Z}$ would look like Figure \ref{fig:degdld}, where the gray region represents the possibilities.

\end{frame}

\begin{frame}{Applications - Degree versus Dimension of Vector Space}
	
	\begin{figure}
		\centering
		\begin{tikzpicture}[scale=0.8]
			\draw[->] (-3.2, 0) -- (7.2, 0) node[right] {$\deg(D)$};
			\draw[->] (0, -1.5) -- (0, 5.2) node[above] {$l(D)$};
			\filldraw[gray!20] (-1,0) -- (2,0) -- (5,3) -- (2,3) -- (-1,0);
			\draw[scale=1.0, domain=2:5, smooth, variable=\x, blue] plot ({\x}, {\x-2});
			\draw[scale=1.0, domain=-1:2, smooth, variable=\x, blue] plot ({\x}, {\x+1});
			\draw[scale=1.0, domain=2:5, smooth, variable=\x, blue] plot ({\x}, {3});
			\draw[scale=1.0, domain=-1:2, smooth, variable=\x, blue] plot ({\x}, {0});
			\filldraw[black] (-1,0) circle (2pt) node[anchor=north]{$(-1,0)$};
			\filldraw[black] (-2,0) circle (2pt) node[anchor=west]{};
			\filldraw[black] (-3,0) circle (2pt) node[anchor=west]{};
			\filldraw[black] (5,3) circle (2pt) node[anchor=north west]{$(2g-1,g)$};
			\filldraw[black] (6,4) circle (2pt) node[anchor=west]{};
			\filldraw[black] (7,5) circle (2pt) node[anchor=west]{};
			\draw[] (2,0) circle (2pt) node[anchor=north]{$(g-1,0)$};
			\draw[] (2,3) circle (2pt) node[anchor=south]{$(g-1,g)$};
			\draw[dashed] (5,3) -- (7.2,5.2);
		\end{tikzpicture}
		\caption{Graph of $\deg(D)$ versus $l(D)$ for a curve with genus $g$}
		\label<1>{fig:degdld}
	\end{figure}

\end{frame}

\begin{frame}{Applications - Group Structure of Curves}
	
	\begin{theorem}[Group Structure of Curves]\label<1>{thm:group}
		Let $\curveC$ be a non-singular projective curve of degree $3$ in $\projective_2$. Let $O$ be an inflection point of $\curveC$. There is a unique additive group structure on $\curveC$ such that $O$ is the zero element and for any $P,Q,R\in\curveC$,
		$$P+Q+R=0$$
		if and only if $P$, $Q$, $R$ are the three points of intersection of a line with $\curveC$.
	\end{theorem}

\end{frame}

\begin{frame}{Applications - Proof of the Existence of a Group Structure of Curves}
	
	$-O=O$ and for $P\neq O$, $-P$ is the other intersection of the line through $P$ and $O$. And $P+Q=-R$ where $R$ is the other intersection of the line through $P$ and $Q$. Therefore, if there is a structure, it is unique. Commutativity is inherent from the definition of the operation.
	
	For the associativity, let
	\begin{align*}
		A &= P+Q\\
		B &= A+R\\
		C &= Q+R\\
		D &= P+C
	\end{align*}
	and show that $B=D$.

\end{frame}

\begin{frame}{Applications - Proof of the Existence of a Group Structure of Curves}
	
	There is a linear polynomial that vanishes at $P$, $Q$, $-A$. The ratio of this polynomial to the polynomial that vanishes at $A$, $-A$, $O$ defines a meromorphic function $\phi_A$ with zeroes at $P$, $Q$ and poles at $A$, $O$. Similarly, $\phi_B$ can be obtained with zeroes at $A$, $R$ and poles at $B$, $O$. Then, $\phi_A\phi_B$ is a meromorphic function with zeroes at $P$, $Q$, $R$ and poles at $B$, $O$, $O$. Doing this for $C$ and $D$, one obtains a meromorphic function $\phi_C\phi_D$ with zeroes at $P$, $Q$, $R$ and poles at $D$, $O$, $O$. The ratio of $\phi_C\phi_D$ and $\phi_A\phi_B$ has a simple zero at $D$ and a simple pole at $B$. By the Riemann-Roch theorem  (\ref{thm:rr}),
	$$l(B)-l(\kappa-B)=\deg(B)-g+1$$
	$l(\kappa-B)=0$ since $\deg(\kappa-B)=\deg(\kappa)-\deg(B)=-1<0$. So, $l(B)=1$. That is, the only meromorphic functions on $\curveC$ with at most a simple pole at $B$ are the constant functions. Therefore, $B=D$.

\end{frame}

\begin{frame}{Applications - Group Structure of Curves}
	
	\begin{figure}
		\centering
		\begin{tikzpicture}[scale=1.0]
			\draw[->] (-2.3, 0) -- (3.4, 0) node[right] {$x$};
			\draw[->] (0, -3.0) -- (0, 3.0) node[above] {$y$};
			\draw[domain=-1.987076:3.1,samples=100,smooth,variable=\x,blue] plot ({\x},{sqrt((\x^3)/3.375 - (\x)/1.5 + 1)});
			\draw[domain=-1.987076:3.1,samples=100,smooth,variable=\x,blue] plot ({\x},{-sqrt((\x^3)/3.375 - (\x)/1.5 + 1)});
			\node (P) at (-1.95, 0.321) {};
			\node (Q) at (0.6, 0.81486) {};
			\node (R) at (0.16233, -0.945) {};
			\node (NPR) at (3.0, -2.64575) {};
			\node (PR) at (3.0, 2.64575) {};
			\node (NPQ) at (1.47657, 0.98463) {};
			\node (PQ) at (1.47657, -0.98463) {};
			\node (NPQR) at (-1.63584, -0.8908) {};
			\filldraw[black] (P) circle (1pt) node[anchor=east]{$P$};
			\filldraw[black] (Q) circle (1pt) node[anchor=south]{$Q$};
			\filldraw[black] (R) circle (1pt) node[anchor=north]{$R$};
			\filldraw[black] (NPR) circle (1pt) node[anchor=west]{$-(P+R)$};
			\filldraw[black] (PR) circle (1pt) node[anchor=west]{$P+R$};
			\filldraw[black] (NPQ) circle (1pt) node[anchor=west]{$-(P+Q)$};
			\filldraw[black] (PQ) circle (1pt) node[anchor=west]{$P+Q$};
			\filldraw[black] (NPQR) circle (1pt) node[anchor=east]{$-(P+Q+R)$};
			\draw[dashed] (P) -- (NPQ);
			\draw[dashed] (P) -- (NPR);
			\draw[dashed, color=orange] (NPQ) -- (PQ);
			\draw[dashed, color=orange] (NPR) -- (PR);
			\draw[dashed] (NPQR) -- (PQ);
			\draw[dashed] (NPQR) -- (PR);
		\end{tikzpicture}
		\caption{Associativity of the elliptic curve group operation}
		\label<1>{fig:ec}
	\end{figure}

\end{frame}

\begin{frame}{Applications}

	\begin{remark}
		Theorem \ref{thm:group}, for a finite field instead of $\CC$, is what makes elliptic curve cryptography possible, which is an integral part of the modern cryptology.
	\end{remark}

\end{frame}

% \begin{frame}
% 	Frame without a title
% 	\begin{alertblock}{Important theorem}
% 		Sample text in red box
% 	\end{alertblock}
% \end{frame}

% \begin{frame}{Blocks}
% 	Three different block environments are pre-defined and may be styled with an
% 	optional background color. % \cite{Knuth92}
	
% 	\begin{columns}[T,onlytextwidth]
% 		\column{0.45\textwidth}
% 		\begin{block}{Default}
% 			Block content.
% 		\end{block}
		
% 		\begin{alertblock}{Alert}
% 			Block content.
% 		\end{alertblock}
		
% 		\begin{exampleblock}{Example}
% 			Block content.
% 		\end{exampleblock}
		
% 		\column{0.45\textwidth}
		
% 		\metroset{block=fill}
		
% 		\begin{block}{Default}
% 			Block content.
% 		\end{block}
		
% 		\begin{alertblock}{Alert}
% 			Block content.
% 		\end{alertblock}
		
% 		\begin{exampleblock}{Example}
% 			Block content.
% 		\end{exampleblock}
		
% 	\end{columns}
% \end{frame}

\appendix

\begin{frame}[allowframebreaks]{References}
	\begin{thebibliography}{9}
    \bibitem{ref:miranda} \textbf{Algebraic Curves and Riemann Surfaces}, by Rick Miranda.
    \bibitem{ref:hartshorne} \textbf{Algebraic Geometry}, by Robert Hartshorne.
    \bibitem{ref:kirwan} \textbf{Complex Algebraic Curves}, by Frances Kirwan.
    \bibitem{ref:fulton} \textbf{Algebraic Curves}, by William Fulton.
    \bibitem{ref:keith} \textbf{Elementary Algebraic Geometry}, by Keith Kendig.
    \bibitem{ref:hampus} \textbf{Elementary Proof of Riemann-Roch Theorem}, by Hampus Sundgren.
    \bibitem{ref:terrytao} \textbf{\href{https://terrytao.wordpress.com/2018/03/28/246c-notes-1-meromorphic-functions-on-riemann-surfaces-and-the-riemann-roch-theorem/}{Blog of Terence Tao}}
\end{thebibliography}

\end{frame}


\begin{frame}{Table of contents}
	\setbeamertemplate{section in toc}[sections numbered]
	\tableofcontents%[hideallsubsections]
\end{frame}

\section{Introduction}

\begin{frame}{Introduction}

    We will look at geometrically finite Fuchsian groups and the Poincare theorem. First, some notions such as cycles and periods of the fundamental domains of Fuchsian groups are defined. Then, we look at what being geometrically finite means for a Fuchsian group. Finally, we define the signatures, which compactly represent the Fuchsian groups, and we see that almost all signatures are possible to be attained by a Fuchsian group.

	\begin{exampleblock}{}
	  {\large ``It is by logic that we prove, but by intuition that we discover. To know how to criticize is good, to know how to create is better.''}
	  \vskip3mm
	  \hspace*\fill{\small--- Poincare}
	\end{exampleblock}

\end{frame}

\begin{frame}{Introduction - Poincare Theorem}

    The main theorem of this presentation:

	\begin{theorem}[Poincare]
        There exists a Fuchsian group with signature $(g;m_1,m_2,\cdots,m_r;s)$ if $g\geq 0$, $r\geq 0$, $m_i\geq 2$, $s\geq 0$ are integers and $$(2g-2)+\sum_{i=1}^r\left(1-\frac{1}{m_i}\right)+s>0$$
    \end{theorem}

\end{frame}

\section{The Structure of Dirichlet Domains}

\begin{frame}{Sides, Vertices, Faces}

	Dirichlet domains are bounded by geodesics and possibly segments of real axis. These bounding geodesic segments are called $\vocab{sides}$, and the intersection of them, together with elliptic points of order 2 on the segments, are called $\vocab{vertices}$. Tessellation of $\mathbb{H}$ by a Dirichlet domain $F$, $\{T(F):T\in\Gamma\}$, consists of the images of $F$ under $\Gamma$, which are called \vocab{faces}.

\end{frame}

\begin{frame}{Congruent Cycles and Periods}

	Two points $u,v\in\mathbb{H}$ are called \vocab{congruent} if they belong to the same orbit of $\Gamma$. The congruence is an equivalence relation on the vertices of a Dirichlet domain and the equivalence classes are called \vocab{cycles}. If $u$ is fixed by an elliptic element $S$, then $v=Tu$ is fixed by $TST^{-1}$. Thus, if one vertex of a cycle is fixed by an elliptic element, then all vertices of that cycle are fixed by conjugate elliptic elements. Such a cycle is called \vocab{elliptic cycle}, vertices are called \vocab{elliptic vertices}, and the number of elliptic cycles is equal to the non-congruent elliptic points in $F$. If a point $w\in\H$ has a non-trivial stabilizer, then this stabilizer is a maximal finite cyclic subgroup of $\Gamma$. The order of non-conjugate maximal finite cyclic subgroups of $\Gamma$ are called \vocab{periods}.

\end{frame}

\begin{frame}{The Structure of Dirichlet Domains}

    \begin{remark}
        There is a one-to-one correspondence between the elliptic cycles of $F$ and the conjugacy classes of non-trivial maximal finite cyclic subgroups of $\Gamma$.
    \end{remark}

\end{frame}

\begin{frame}{Example}

    Take $\Gamma=\PSL_2(\mathbb{Z})$, the modular group, and take $F$ to be the fundamental domain below.

    \begin{minipage}{0.24\textwidth}
		\begin{center}
        \begin{tikzpicture}[scale=1.2]
        \draw[very thick,fill=gray!30] (0.5, 3.0) -- (0.5, 0.8660254037844386) arc (59.99999999999999:120.00000000000001:0.9999999999999999) -- (-0.5, 3.0);
        
        \draw[-latex] (-0.7-0.5,0) -- (0.7+0.5,0)node[below]{Re};
        \draw[-latex] (0,0) -- (0,3.0+0.5)node[right]{Im};
        \path(-1,0) --node[below, pos=0]{$-1$}node[below right, pos=.5]{0}node[below, pos=1]{1} (1,0) (0,1)node[below right]{$i$};
        \end{tikzpicture}
        \end{center}
	\end{minipage}
	\begin{minipage}{0.75\textwidth}
		Vertices of $F$ are $w,w-1,i,\infty$ where $w=\frac{1}{2}+\frac{\sqrt{3}}{2}i$. Any point $u$ on the left side is congruent to $u+1$ on the right side via $z\mapsto z+1$. $w,w-1,i$ are fixed by the cyclic finite groups generated by $z\mapsto\frac{z-1}{z}$, $z\mapsto\frac{-z-1}{z}$, and $z\mapsto\frac{-1}{z}$ respectively. $w$ and $w-1$ are congruent vertices, so $\{w,w-1\}$ and $\{i\}$ are the elliptic cycles. Non-conjugate maximal finite cyclic subgroups of $\Gamma$ are $\{1,S\}$ and $\{1,U,U^2\}$ where $S:z\mapsto \frac{-1}{z}$, $T:z\mapsto z+1$, and $U=ST:z\mapsto \frac{-1}{z+1}$. A parabolic element can be considered as an infinite order elliptic element, hence the stabilizer of an element $w\in\overline{\mathbb{R}}$ is a maximal cyclic parabolic subgroup. If we allow infinite periods, then since $\{1,T,T^2,\cdots,\}$ is a maximal cyclic parabolic subgroup of $\Gamma$, the modular group has periods $2,3,\infty$.
	\end{minipage}
    
\end{frame}

\begin{frame}{Angle Sum of Congruent Cycle}

	\begin{theorem}\label<1>{thm:2pim}
    Let $F$ be a Dirichlet domain for $\Gamma$. Let $\theta_1,\theta_2,\cdots,\theta_t$ be the internal angles at a cycle of $F$ (there are finitely many points in a congruent cycle because $F$ is locally finite). Let $m$ be the order of the stabilizer of one of these vertices (stabilizers of two points in a congruent cycle are conjugate subgroups of $\Gamma$, so they have the same order). Then $$\theta_1+\theta_2+\cdots+\theta_t=\frac{2\pi}{m}$$
    \end{theorem}

\end{frame}

\begin{frame}{Angle Sum of Congruent Cycle - Proof}

	Let $v_1,v_2,\cdots,v_t$ be the vertices of the congruent cycle, $\theta_1,\theta_2,\cdots,\theta_t$ be the angles, and $$H=\{\text{Id},S,S^2,\cdots,S^{m-1}\}$$ be the stabilizer of $v_1$. Each $S^r(F)$ has a vertex at $S^r(v_1)=v_1$ with angle $\theta_1$. Now we will look at other vertices being sent to $v_1$. Suppose $T_k(v_k)=v_1$, then each $S^rT_k(F)$ has a vertex at $S^rT_k(v_k)=v_1$ with angle $\theta_k$. We have $mt$ angles surrounding $v_1$, which add up to $2\pi$.

\end{frame}

\begin{frame}{Angle Sum of Congruent Cycle - Example}

    \begin{minipage}{0.49\textwidth}
		\begin{center}
        \begin{tikzpicture}[scale=1.6]
        \draw[very thick,fill=gray!30] (0.5, 3.0) -- (0.5, 0.8660254037844386) arc (59.99999999999999:120.00000000000001:0.9999999999999999) -- (-0.5, 3.0);
        
        \draw[-latex] (-0.7-0.5,0) -- (0.7+0.5,0)node[below]{Re};
        \draw[-latex] (0,0) -- (0,3.0+0.5)node[right]{Im};
        \path(-1,0) --node[below, pos=0]{$-1$}node[below right, pos=.5]{0}node[below, pos=1]{1} (1,0) (0,1)node[below right]{$i$};
        \end{tikzpicture}
        \end{center}
	\end{minipage}
	\begin{minipage}{0.5\textwidth}
		In the modular group, $\{w,w-1\}$ is a congruent cycle, and the sum of angles at these vertices is $\frac{\pi}{3}+\frac{\pi}{3}=\frac{2\pi}{3}$ as $m=3$. $\{i\}$ is a congruent cycle, and the angle is $\pi=\frac{2\pi}{2}$ as $m=2$.
	\end{minipage}

\end{frame}

\begin{frame}{Pairing the Sides}

	Sides can also be congruent. For a side $s$ and $T\in\Gamma$, if $T(s)$ is a side too, then $s$ and $T(s)$ are called \vocab{congruent sides}. Sides of $F$ fall into congruent pairs.

    \begin{theorem}\label<1>{thm:genset}
    The subset of $\Gamma$ consisting of elements pairing the sides of $F$ is a generator set for $\Gamma$.
    \end{theorem}

\end{frame}

\begin{frame}{Pairing the Sides - Proof}

	Let $\Gamma'$ be the group generated by the elements pairing the sides of $F$. Take $S\in\Gamma'$. For $U,V\in\Gamma$, such that $S(F)$ and $U(F)$ share a side, and $S(F)$ and $V(F)$ share a vertex, $S^{-1}U$ shares a side with $F$ and $S^{-1}V$ shares a vertex with $F$. $S^{-1}U\in\Gamma'\implies U\in\Gamma'$ and we can go to $S^{-1}V$ from $F$ by following finitely many faces sharing a side, which means $S^{-1}V\in\Gamma'\implies V\in\Gamma'$. Now we have $U,V\in\Gamma'$, so $X=\bigcup_{T\in\Gamma'}T(F)$ and $Y=\bigcup_{T\in\Gamma-\Gamma'}T(F)$ are disjoint and $X\cup Y=\H$. Any union of faces of the tessellation is closed since $F$ is locally finite, and $\H$ is connected. Therefore, $X=\H$ and $Y=\emptyset$.

\end{frame}

\begin{frame}{Pairing the Sides - Example}

    \begin{minipage}{0.49\textwidth}
		\begin{center}
        \begin{tikzpicture}[scale=1.6]
        \draw[very thick,fill=gray!30] (0.5, 3.0) -- (0.5, 0.8660254037844386) arc (59.99999999999999:120.00000000000001:0.9999999999999999) -- (-0.5, 3.0);
        
        \draw[-latex] (-0.7-0.5,0) -- (0.7+0.5,0)node[below]{Re};
        \draw[-latex] (0,0) -- (0,3.0+0.5)node[right]{Im};
        \path(-1,0) --node[below, pos=0]{$-1$}node[below right, pos=.5]{0}node[below, pos=1]{1} (1,0) (0,1)node[below right]{$i$};
        \end{tikzpicture}
        \end{center}
	\end{minipage}
	\begin{minipage}{0.5\textwidth}
		In the modular group, the sides $$\{\frac{1}{2}+ci:c^2\geq\frac{3}{4}\}\text{ and }\{\frac{-1}{2}+ci:c^2\geq\frac{3}{4}\}$$ are congruent via $z\mapsto z+1$. The sides $\{(\cos(\alpha),\sin(\alpha):\frac{\pi}{3}\leq\alpha\leq\frac{\pi}{2})\}$ and $\{(\cos(\alpha),\sin(\alpha):\frac{\pi}{2}\leq\alpha\leq\frac{2\pi}{3})\}$ are congruent via $z\mapsto \frac{-1}{z}$. So, the modular group is generated by $\{z\mapsto z+1,z\mapsto \frac{-1}{z}\}$.
	\end{minipage}

\end{frame}

\begin{frame}{Orbifold}

    Let $\Gamma$ be a Fuchsian group with $\mu(\H/\Gamma)<\infty$ and $F$ be a fundamental domain for it. The restriction of $\pi:\H\rightarrow\H/\Gamma$, $z\mapsto$ the orbit of $z$, to $F$, identifies the congruent points of $F$ that necessarily belong to $\partial F$, and makes $\H/\Gamma$ an oriented surface with possibly some marked points, corresponding to elliptic cycles, and cusps, corresponding to non-congruent vertices at infinity, and this is an \vocab{orbifold}.
    
\end{frame}

\section{Geometry of Fuchsian Groups}

\begin{frame}{Geometrically Finiteness}

	A Fuchsian group is called \vocab{geometrically finite} if there exists a convex fundamental domain for $\Gamma$ with finitely many sides.

    % \begin{exampleblock}
    %     a
    % \end{exampleblock}

    \begin{theorem}[Siegel, \cite{ref:katok} (Katok) 4.1.1]
    A Fuchsian group $\Gamma$ with a finite area is geometrically finite.
    \end{theorem}

\end{frame}

\begin{frame}{Cocompactness}

	A Fuchsian group $\Gamma$ is called \vocab{cocompact} if equivalently one of the following is true:
    \begin{itemize}
        \item $\H/\Gamma$ is compact
        \item any Dirichlet domain $F$ for $\Gamma$ is compact
        \item $\mu(\H/\Gamma)$ is finite and $\Gamma$ does not contain parabolic elements
    \end{itemize}

    \begin{remark}
    There is a one-to-one correspondence between non-congruent vertices at infinity of a Dirichlet fundamental domain for a non-cocompact Fuchsian group $\Gamma$ with finite $\mu(\H/\Gamma)$ and conjugacy classes of maximal parabolic subgroups of $\Gamma$.
    \end{remark}

    Compact fundamental domains have finitely many sides. Non-compact fundamental domains with finite area have at least one vertex at infinity.

\end{frame}

\section{The Signature of a Fuchsian Group and Poincare Theorem}

\begin{frame}{The Signature of a Fuchsian Group}

	For one case, assume that $\Gamma$ is cocompact. It has finitely many sides, vertices, elliptic cycles, and periods $m_1,m_2,\cdots,m_r$. Also, $\H/\Gamma$ is an orbifold, a compact oriented surface of genus $g$. In this case, we say $\Gamma$ has a \vocab{signature} $(g;m_1,m_2,\cdots,m_r)$.

    For the other case, assume that $\Gamma$ is non-cocompact. Assume that $\Gamma$ has $r$ conjugacy classes of maximal elliptic cyclic subgroups of order $m_1,m_2,\cdots,m_r$ and has $s$ conjugacy classes of maximal parabolic cyclic subgroups. Also, $\H/\Gamma$ is an orbifold with genus $g$. In this case, we say $\Gamma$ has a \vocab{signature} $(g;m_1,m_2,\cdots,m_r;s)$. $s=0$ can be considered as the first case.

\end{frame}

\begin{frame}{Area from the Signature}

	\begin{theorem}\label{thm:sigarea}
    Let $\Gamma$ has signature $(g;m_1,m_2,\cdots,m_r;s)$. Then 
    $$\mu(\H/\Gamma)=2\pi\left((2g-2)+\sum_{i=1}^r\left(1-\frac{1}{m_i}\right)+s\right)$$
    In the cocompact case where $s=0$, we have
    $$\mu(\H/\Gamma)=2\pi\left((2g-2)+\sum_{i=1}^r\left(1-\frac{1}{m_i}\right)\right)$$
    \end{theorem}

\end{frame}

\begin{frame}{Area from the Signature - Proof}

	$\mu(\H/\Gamma)=\mu(F)$ and $F$ has $r$ elliptic cycles of vertices. The sum of the angles at all elliptic vertices is $\sum_{i=1}^r\frac{2\pi}{m_i}$ by Theorem \ref{thm:2pim}. There exist $s$ parabolic cycles and $t$ other cycles of vertices. The order of the stabilizer of these vertices is $\infty$ and $1$ respectively, so the sum of the angles at those vertices is $\sum_{i=1}^s\frac{2\pi}{\infty}+\sum_{i=1}^t\frac{2\pi}{1}=2\pi t$. Hence, the sum of all angles of $F$ is $$2\pi\left(t+\sum_{i=1}^r\frac{1}{m_i}\right)$$
    The sides of $F$ are matched up by the elements of $\Gamma$. If we identify these sides, we get an orbifold of a genus $g$. It has $r+s+t$ vertices, $1$ face, and $n$ edges, where $n$ is the number of sets of identified sides. By the Euler formula, $$2-2g=\chi=(r+s+t)-n+1$$

\end{frame}

\begin{frame}{Gauss-Bonnet Hyperbolic Polygon Area Formula}

	\begin{lemma}[Gauss-Bonnet]
        A $n$ sided hyperbolic polygon $P$ with angles $\alpha_1,\alpha_2,\cdots,\alpha_n$ has area $$\mu(P)=(n-2)\pi+\sum_{i=1}^n\alpha_i$$
        To prove it, simply divide the polygon into triangles and use $\mu(\triangle)=\pi-\alpha-\beta-\gamma$.
    \end{lemma}

\end{frame}

\begin{frame}{Area from the Signature - Proof}

	$F$ has $2n$ sides, $2$ for every matched up set. By the Gauss-Bonnet formula, we have 
    \begin{align*}
        \mu(F) &= (2n-2)\pi-2\pi\left(t+\sum_{i=1}^r\frac{1}{m_i}\right)\\
        &= 2\pi\left((n-1)-t-\sum_{i=1}^r\frac{1}{m_i}\right)\\
        &= 2\pi\left((n-1)-(1-2g+n-r-s)-\sum_{i=1}^r\frac{1}{m_i}\right)\\
        &= 2\pi\left((2g-2)+\sum_{i=1}^r \left(1-\frac{1}{m_i}\right)+s\right)
    \end{align*}

\end{frame}

\begin{frame}{Area from the Signature - Example}

    \begin{minipage}{0.39\textwidth}
		\begin{center}
        \begin{tikzpicture}[scale=1.6]
        \draw[very thick,fill=gray!30] (0.5, 3.0) -- (0.5, 0.8660254037844386) arc (59.99999999999999:120.00000000000001:0.9999999999999999) -- (-0.5, 3.0);
        
        \draw[-latex] (-0.7-0.5,0) -- (0.7+0.5,0)node[below]{Re};
        \draw[-latex] (0,0) -- (0,3.0+0.5)node[right]{Im};
        \path(-1,0) --node[below, pos=0]{$-1$}node[below right, pos=.5]{0}node[below, pos=1]{1} (1,0) (0,1)node[below right]{$i$};
        \end{tikzpicture}
        \end{center}
	\end{minipage}
	\begin{minipage}{0.6\textwidth}
        The modular group has signature $(0;2,3;1)$, so 
        $$\mu(\H/\PSL_2(\mathbb{Z}))=2\pi\left(-2+\frac{1}{2}+\frac{2}{3}+1\right)=\frac{\pi}{3}$$
	\end{minipage}

\end{frame}

\begin{frame}{Poincare Theorem}

	Are all signatures possible? It is not, however, the only restriction is to get a positive area from the formula in Theorem \ref{thm:sigarea}.

    \begin{theorem}[Poincare]\label<1>{thm:poincare}
        There exists a Fuchsian group with signature $(g;m_1,m_2,\cdots,m_r;s)$ if $g\geq 0$, $r\geq 0$, $m_i\geq 2$, $s\geq 0$ are integers and $$(2g-2)+\sum_{i=1}^r\left(1-\frac{1}{m_i}\right)+s>0$$
    \end{theorem}

\end{frame}

\begin{frame}{Poincare Theorem - Proof}

	Take a regular $4g+r+s$ sided hyperbolic polygon in unit disc model where the vertices are $0<t<1$ Euclidean distance from the center. On the first $r$ sides, construct external isosceles hyperbolic triangles with apex angle $\frac{2\pi}{m_i}$ (in case $m_i$ is $2$, we take the midpoint of the base as the apex vertex) and on the next $s$ sides, construct isosceles hyperbolic triangles with apex angle $0$ to turn it into a polygon $P(t)$ with $4g+2r+2s$ vertices. The Figure \ref{fig:fds} depicts an example for $g=2$, $r=3$, $s=1$.

\end{frame}

\begin{frame}{The Signature of a Fuchsian Group - Proof}

	\begin{figure}
    \centering
    \begin{tikzpicture}[scale=3.0]
        \draw (0,0) circle (1);
        \clip (0,0) circle (1);
        \hgline{0}{180}
        \hgline{30}{210}
        \hgline{60}{240}
        \hgline{90}{270}
        \hgline{120}{300}
        \hgline{150}{330}
        \begin{scope}
            \clip (0,0) circle (0.7);
            \hgline{\anglex+30*0}{-\anglex+30*0-30}
            \hgline{\anglex+30*1}{-\anglex+30*1-30}
            \hgline{\anglex+30*2}{-\anglex+30*2-30}
            \hgline{\anglex+30*3}{-\anglex+30*3-30}
            \hgline{\anglex+30*4}{-\anglex+30*4-30}
            \hgline{\anglex+30*5}{-\anglex+30*5-30}
            \hgline{\anglex+30*6}{-\anglex+30*6-30}
            \hgline{\anglex+30*7}{-\anglex+30*7-30}
            \hgline{\anglex+30*8}{-\anglex+30*8-30}
            \hgline{\anglex+30*9}{-\anglex+30*9-30}
            \hgline{\anglex+30*10}{-\anglex+30*10-30}
            \hgline{\anglex+30*11}{-\anglex+30*11-30}
        \end{scope}
    
        \node[label=$v_{1}$] at ({0.7 * cos(30*1-30)}, {-0.7 * sin(30*1-30)})[circle,fill,inner sep=1.0pt]{};
        \node[label=$v_{4}$] at ({0.7 * cos(30*2-30)}, {-0.7 * sin(30*2-30)})[circle,fill,inner sep=1.0pt]{};
        \node[label=$v_{3}$] at ({0.7 * cos(30*3-30)}, {-0.7 * sin(30*3-30)})[circle,fill,inner sep=1.0pt]{};
        \node[label=$v_{2}$] at ({0.7 * cos(30*4-30)}, {-0.7 * sin(30*4-30)})[circle,fill,inner sep=1.0pt]{};
        \node[label=$v_{5}$] at ({0.7 * cos(30*5-30)}, {-0.7 * sin(30*5-30)})[circle,fill,inner sep=1.0pt]{};
        \node[label=$v_{8}$] at ({0.7 * cos(30*6-30)}, {-0.7 * sin(30*6-30)})[circle,fill,inner sep=1.0pt]{};
        \node[label=$v_{7}$] at ({0.7 * cos(30*7-30)}, {-0.7 * sin(30*7-30)})[circle,fill,inner sep=1.0pt]{};
        \node[label=$v_{6}$] at ({0.7 * cos(30*8-30)}, {-0.7 * sin(30*8-30)})[circle,fill,inner sep=1.0pt]{};
        \node[label=$v_{9}$] at ({0.7 * cos(30*9-30)}, {-0.7 * sin(30*9-30)})[circle,fill,inner sep=1.0pt]{};
        \node[label=$v_{10}$] at ({0.7 * cos(30*10-30)}, {-0.7 * sin(30*10-30)})[circle,fill,inner sep=1.0pt]{};
        \node[label=$v_{11}$] at ({0.7 * cos(30*11-30)}, {-0.7 * sin(30*11-30)})[circle,fill,inner sep=1.0pt]{};
        \node[label=$v_{12}$] at ({0.7 * cos(30*12-30)}, {-0.7 * sin(30*12-30)})[circle,fill,inner sep=1.0pt]{};
        
        \node[label=$w_1$] at ({0.8 * cos(15)}, {0.8 * sin(15)})[circle,fill,inner sep=1.0pt]{};
        \begin{scope}
            \clip (0,0) circle (0.8);
            \clip ({0.8 * cos(15)}, {0.8 * sin(15)}) circle (0.22);
            \hgline{10}{50}
            \hgline{-20}{20}
        \end{scope}
        \node[label=$w_2$] at ({0.635 * cos(45)}, {0.635 * sin(45)})[circle,fill,inner sep=1.0pt]{};
        \node[label=$w_3$] at ({0.9 * cos(75)}, {0.9 * sin(75)})[circle,fill,inner sep=1.0pt]{};
        \begin{scope}
            \clip (0,0) circle (0.9);
            \clip ({0.9 * cos(75)}, {0.9 * sin(75)}) circle (0.28);
            \hgline{74}{115}
            \hgline{35}{76}
        \end{scope}
        \node[label=$w_4$] at ({1.0 * cos(105)}, {1.0 * sin(105)})[circle,fill,inner sep=1.0pt]{};
        \begin{scope}
            \clip ({1.0 * cos(105)}, {1.0 * sin(105)}) circle (0.37);
            \hgline{105}{147}
            \hgline{63}{105}
        \end{scope}
        
        \draw[-Stealth] (-0.14+0.005,0.776-0.004) -- (-0.14,0.776) node[above]{\footnotesize $\xi_4'$};
        \draw[-Stealth] (-0.267-0.001,0.749-0.004) -- (-0.267,0.749) node[above left]{\footnotesize $\xi_4$};
    
        \draw[-Stealth] (0.14-0.005,0.767-0.004) -- (0.14,0.767) node[above]{\footnotesize $\lambda_3'$};
        \draw[-Stealth] (0.26+0.001,0.749-0.004) -- (0.26,0.749) node[right]{\footnotesize $\lambda_3$};
    
        \draw[-Stealth] (0.4-0.005,0.51+0.007) -- (0.4,0.51) node[below]{\footnotesize $\lambda_2'$};
        \draw[-Stealth] (0.51+0.007,0.4-0.005) -- (0.51,0.4) node[below]{\footnotesize $\lambda_2$};
    
        \draw[-Stealth] (0.684-0.005,0.25+0.007) -- (0.684,0.25) node[below]{\footnotesize $\lambda_1'$};
        \draw[-Stealth] (0.726-0.001,0.125-0.004) -- (0.726,0.125) node[right]{\footnotesize $\lambda_1$};
        
        \draw[-Stealth] ({0.639*cos(15+30*4)+0.01*sin(15+30*4)},{0.639*sin(15+30*4)-0.01*cos(15+30*4)}) -- ({0.639*cos(15+30*4)},{0.639*sin(15+30*4)}) node[right]{\footnotesize $\alpha_1$};
        \draw[-Stealth] ({0.639*cos(15+30*5)+0.01*sin(15+30*5)},{0.639*sin(15+30*5)-0.01*cos(15+30*5)}) -- ({0.639*cos(15+30*5)},{0.639*sin(15+30*5)}) node[right]{\footnotesize $\beta_1'$};
        \draw[-Stealth] ({0.639*cos(15+30*8)+0.01*sin(15+30*8)},{0.639*sin(15+30*8)-0.01*cos(15+30*8)}) -- ({0.639*cos(15+30*8)},{0.639*sin(15+30*8)}) node[above]{\footnotesize $\alpha_2$};
        \draw[-Stealth] ({0.639*cos(15+30*9)+0.01*sin(15+30*9)},{0.639*sin(15+30*9)-0.01*cos(15+30*9)}) -- ({0.639*cos(15+30*9)},{0.639*sin(15+30*9)}) node[above]{\footnotesize $\beta_2'$};
    
        \draw[-Stealth] ({0.639*cos(15+30*6)-0.01*sin(15+30*6)},{0.639*sin(15+30*6)+0.01*cos(15+30*6)}) -- ({0.639*cos(15+30*6)},{0.639*sin(15+30*6)}) node[right]{\footnotesize $\alpha_1'$};
        \draw[-Stealth] ({0.639*cos(15+30*7)-0.01*sin(15+30*7)},{0.639*sin(15+30*7)+0.01*cos(15+30*7)}) -- ({0.639*cos(15+30*7)},{0.639*sin(15+30*7)}) node[above]{\footnotesize $\beta_1$};
        \draw[-Stealth] ({0.639*cos(15+30*10)-0.01*sin(15+30*10)},{0.639*sin(15+30*10)+0.01*cos(15+30*10)}) -- ({0.639*cos(15+30*10)},{0.639*sin(15+30*10)}) node[right]{\footnotesize $\alpha_2'$};
        \draw[-Stealth] ({0.639*cos(15+30*11)-0.01*sin(15+30*11)},{0.639*sin(15+30*11)+0.01*cos(15+30*11)}) -- ({0.639*cos(15+30*11)},{0.639*sin(15+30*11)}) node[right]{\footnotesize $\beta_2$};
    \end{tikzpicture}
    \caption{The polygon $P(t)$}
    \label<1>{fig:fds}
    \end{figure}

\end{frame}

\begin{frame}[plain]

	\begin{figure}
    \centering
    \begin{tikzpicture}[scale=5.0]
        \draw (0,0) circle (1);
        \clip (0,0) circle (1);
        \hgline{0}{180}
        \hgline{30}{210}
        \hgline{60}{240}
        \hgline{90}{270}
        \hgline{120}{300}
        \hgline{150}{330}
        \begin{scope}
            \clip (0,0) circle (0.7);
            \hgline{\anglex+30*0}{-\anglex+30*0-30}
            \hgline{\anglex+30*1}{-\anglex+30*1-30}
            \hgline{\anglex+30*2}{-\anglex+30*2-30}
            \hgline{\anglex+30*3}{-\anglex+30*3-30}
            \hgline{\anglex+30*4}{-\anglex+30*4-30}
            \hgline{\anglex+30*5}{-\anglex+30*5-30}
            \hgline{\anglex+30*6}{-\anglex+30*6-30}
            \hgline{\anglex+30*7}{-\anglex+30*7-30}
            \hgline{\anglex+30*8}{-\anglex+30*8-30}
            \hgline{\anglex+30*9}{-\anglex+30*9-30}
            \hgline{\anglex+30*10}{-\anglex+30*10-30}
            \hgline{\anglex+30*11}{-\anglex+30*11-30}
        \end{scope}
    
        \node[label=$v_{1}$] at ({0.7 * cos(30*1-30)}, {-0.7 * sin(30*1-30)})[circle,fill,inner sep=1.0pt]{};
        \node[label=$v_{4}$] at ({0.7 * cos(30*2-30)}, {-0.7 * sin(30*2-30)})[circle,fill,inner sep=1.0pt]{};
        \node[label=$v_{3}$] at ({0.7 * cos(30*3-30)}, {-0.7 * sin(30*3-30)})[circle,fill,inner sep=1.0pt]{};
        \node[label=$v_{2}$] at ({0.7 * cos(30*4-30)}, {-0.7 * sin(30*4-30)})[circle,fill,inner sep=1.0pt]{};
        \node[label=$v_{5}$] at ({0.7 * cos(30*5-30)}, {-0.7 * sin(30*5-30)})[circle,fill,inner sep=1.0pt]{};
        \node[label=$v_{8}$] at ({0.7 * cos(30*6-30)}, {-0.7 * sin(30*6-30)})[circle,fill,inner sep=1.0pt]{};
        \node[label=$v_{7}$] at ({0.7 * cos(30*7-30)}, {-0.7 * sin(30*7-30)})[circle,fill,inner sep=1.0pt]{};
        \node[label=$v_{6}$] at ({0.7 * cos(30*8-30)}, {-0.7 * sin(30*8-30)})[circle,fill,inner sep=1.0pt]{};
        \node[label=$v_{9}$] at ({0.7 * cos(30*9-30)}, {-0.7 * sin(30*9-30)})[circle,fill,inner sep=1.0pt]{};
        \node[label=$v_{10}$] at ({0.7 * cos(30*10-30)}, {-0.7 * sin(30*10-30)})[circle,fill,inner sep=1.0pt]{};
        \node[label=$v_{11}$] at ({0.7 * cos(30*11-30)}, {-0.7 * sin(30*11-30)})[circle,fill,inner sep=1.0pt]{};
        \node[label=$v_{12}$] at ({0.7 * cos(30*12-30)}, {-0.7 * sin(30*12-30)})[circle,fill,inner sep=1.0pt]{};
        
        \node[label=$w_1$] at ({0.8 * cos(15)}, {0.8 * sin(15)})[circle,fill,inner sep=1.0pt]{};
        \begin{scope}
            \clip (0,0) circle (0.8);
            \clip ({0.8 * cos(15)}, {0.8 * sin(15)}) circle (0.22);
            \hgline{10}{50}
            \hgline{-20}{20}
        \end{scope}
        \node[label=$w_2$] at ({0.635 * cos(45)}, {0.635 * sin(45)})[circle,fill,inner sep=1.0pt]{};
        \node[label=$w_3$] at ({0.9 * cos(75)}, {0.9 * sin(75)})[circle,fill,inner sep=1.0pt]{};
        \begin{scope}
            \clip (0,0) circle (0.9);
            \clip ({0.9 * cos(75)}, {0.9 * sin(75)}) circle (0.28);
            \hgline{74}{115}
            \hgline{35}{76}
        \end{scope}
        \node[label=$w_4$] at ({1.0 * cos(105)}, {1.0 * sin(105)})[circle,fill,inner sep=1.0pt]{};
        \begin{scope}
            \clip ({1.0 * cos(105)}, {1.0 * sin(105)}) circle (0.37);
            \hgline{105}{147}
            \hgline{63}{105}
        \end{scope}
        
        \draw[-Stealth] (-0.14+0.005,0.776-0.004) -- (-0.14,0.776) node[above]{\footnotesize $\xi_4'$};
        \draw[-Stealth] (-0.267-0.001,0.749-0.004) -- (-0.267,0.749) node[above left]{\footnotesize $\xi_4$};
    
        \draw[-Stealth] (0.14-0.005,0.767-0.004) -- (0.14,0.767) node[above]{\footnotesize $\lambda_3'$};
        \draw[-Stealth] (0.26+0.001,0.749-0.004) -- (0.26,0.749) node[right]{\footnotesize $\lambda_3$};
    
        \draw[-Stealth] (0.4-0.005,0.51+0.007) -- (0.4,0.51) node[below]{\footnotesize $\lambda_2'$};
        \draw[-Stealth] (0.51+0.007,0.4-0.005) -- (0.51,0.4) node[below]{\footnotesize $\lambda_2$};
    
        \draw[-Stealth] (0.684-0.005,0.25+0.007) -- (0.684,0.25) node[below]{\footnotesize $\lambda_1'$};
        \draw[-Stealth] (0.726-0.001,0.125-0.004) -- (0.726,0.125) node[right]{\footnotesize $\lambda_1$};
        
        \draw[-Stealth] ({0.639*cos(15+30*4)+0.01*sin(15+30*4)},{0.639*sin(15+30*4)-0.01*cos(15+30*4)}) -- ({0.639*cos(15+30*4)},{0.639*sin(15+30*4)}) node[right]{\footnotesize $\alpha_1$};
        \draw[-Stealth] ({0.639*cos(15+30*5)+0.01*sin(15+30*5)},{0.639*sin(15+30*5)-0.01*cos(15+30*5)}) -- ({0.639*cos(15+30*5)},{0.639*sin(15+30*5)}) node[right]{\footnotesize $\beta_1'$};
        \draw[-Stealth] ({0.639*cos(15+30*8)+0.01*sin(15+30*8)},{0.639*sin(15+30*8)-0.01*cos(15+30*8)}) -- ({0.639*cos(15+30*8)},{0.639*sin(15+30*8)}) node[above]{\footnotesize $\alpha_2$};
        \draw[-Stealth] ({0.639*cos(15+30*9)+0.01*sin(15+30*9)},{0.639*sin(15+30*9)-0.01*cos(15+30*9)}) -- ({0.639*cos(15+30*9)},{0.639*sin(15+30*9)}) node[above]{\footnotesize $\beta_2'$};
    
        \draw[-Stealth] ({0.639*cos(15+30*6)-0.01*sin(15+30*6)},{0.639*sin(15+30*6)+0.01*cos(15+30*6)}) -- ({0.639*cos(15+30*6)},{0.639*sin(15+30*6)}) node[right]{\footnotesize $\alpha_1'$};
        \draw[-Stealth] ({0.639*cos(15+30*7)-0.01*sin(15+30*7)},{0.639*sin(15+30*7)+0.01*cos(15+30*7)}) -- ({0.639*cos(15+30*7)},{0.639*sin(15+30*7)}) node[above]{\footnotesize $\beta_1$};
        \draw[-Stealth] ({0.639*cos(15+30*10)-0.01*sin(15+30*10)},{0.639*sin(15+30*10)+0.01*cos(15+30*10)}) -- ({0.639*cos(15+30*10)},{0.639*sin(15+30*10)}) node[right]{\footnotesize $\alpha_2'$};
        \draw[-Stealth] ({0.639*cos(15+30*11)-0.01*sin(15+30*11)},{0.639*sin(15+30*11)+0.01*cos(15+30*11)}) -- ({0.639*cos(15+30*11)},{0.639*sin(15+30*11)}) node[right]{\footnotesize $\beta_2$};
    \end{tikzpicture}
    \end{figure}

\end{frame}

\begin{frame}{Poincare Theorem - Proof}

	As $t\rightarrow 0$, $\mu(P(t))\rightarrow 0$. As $t\rightarrow 1$, the angles except for $\frac{2\pi}{m_i}$ vanish, so by the Gauss-Bonnet formula, we have $$\mu(P(t))=(4g+2r+2s-2)\pi-\sum_{i=1}^r\frac{2\pi}{m_i}=2\pi\left((2g-1)+\sum_{i=1}^r\left(1-\frac{1}{m_i}\right)+s\right)$$
    Since $\mu(P(t))$ is continuous, for some $\tilde{t}$ between $0$ and $1$, $\mu(P(\tilde{t}))$ becomes the desired value 
    $$\mu(P(\tilde{t}))=2\pi\left((2g-2)+\sum_{i=1}^r\left(1-\frac{1}{m_i}\right)+s\right)$$
    in Theorem \ref{thm:sigarea}. We take this $P(\tilde{t})$ as our polygon $P$.

\end{frame}

\begin{frame}{Poincare Theorem - Proof}

	For any two geodesics with equal length, there exists an isometry mapping one to other. Take $A_i$, $B_j$, $X_k$, $Y_l$ for $i,j\in\{1,2,\cdots,g\}$, $k\in\{1,2,\cdots,r\}$, $l\in\{1,2,\cdots s\}$ such that
    $$A(\alpha'_i)=\alpha_i,\qquad B(\beta'_j)=\beta_j,\qquad X(\lambda'_k)=\lambda_k,\qquad Y(\xi'_l)=Y(\xi_l)$$

\end{frame}

\begin{frame}{Poincare Theorem - Proof}

	Now, we compute the congruence classes of the vertices. $v_1$ is congruent to $v_2$ via $B_g^{-1}$. This $v_2$ is congruent to $v_3$ via $A_g^{-1}$. This $v_3$ is congruent to $v_4$ via $B_g$. Proceeding with this process, also considering the $r+s$ vertices that we build isosceles triangles on are also congruent via $X_k$ and $Y_l$, so we see that all vertices of the regular polygon that we begin with at the start form a congruent set. So, $$X_1X_2\cdots X_rY_1Y_2\cdots Y_sA_1B_1A_1^{-1}B_1^{-1}\cdots A_gB_gA_g^{-1}B_g^{-1}(v_1)=v_1$$
    The other vertices $w_1,w_2,\cdots,w_r$ form $r$ congruent sets each with just one element.

\end{frame}

\begin{frame}{Poincare Theorem - Proof}

	Let the sum of angles at the congruent set of vertices $v_1,\cdots,v_{4g+r}$ be $\alpha$. Because of our choice of $P$, we have
    $$\mu(P)=2\pi\left((2g-2)+\sum_{i=1}^r\left(1-\frac{1}{m_i}\right)+s\right)$$
    Also, by the Gauss-Bonnet formula again, we have
    \begin{align*}
        \mu(P) &= (4g+2r+2s-2)\pi-\left(\alpha+\sum_{i=1}^r\frac{2\pi}{m_i}\right)\\
        &= 2\pi\left((2g-1)+\sum_{i=1}^r\left(1-\frac{1}{m_i}\right)+s\right)-\alpha
    \end{align*}
    So, $\alpha=2\pi$.

\end{frame}

\begin{frame}{Poincare Theorem - Proof}

	Let $\Gamma$ be the group generated by $A_i$, $B_j$, $X_k$, $Y_l$. By Theorem \ref{thm:genset}, we expect $\Gamma$ to be the group we want. The sum of angles at congruent vertices $v_1,\cdots,v_{4g+r}$ is $2\pi$, the angle at $w_k$ is $\frac{2\pi}{m_k}$, and the other angles are $0$. By Theorem \ref{thm:2pim}, $P$ is a fundamental region for $\Gamma$. $\H/\Gamma$ has $r+s+1$ congruent set of vertices, $2g+r+s$ edges, and $1$ face. By Euler formula, 
    $$2-2g=(r+s+1)-(2g+r+s)+1$$
    we see that it has genus $g$. There are $r$ elliptic cycles, $\{w_1\},\{w_2\},\cdots,\{w_r\}$, and their stabilizers have orders $m_1,m_2,\cdots,m_r$. There are $s$ conjugacy classes of maximal parabolic cyclic subgroups. Hence, $\Gamma$ has signature $(g;m_1,m_2,\cdots,m_r;s)$.

\end{frame}

\begin{frame}{The Representation of the Group}

	The representation of the group $\Gamma$ with signature $(g;m_1,m_2,\cdots,m_r;s)$ is 
    \begin{align*}
    \Gamma=\langle &A_1,\cdots,A_g,B_1,\cdots,B_g,X_1,\cdots,X_r,Y_1,\cdots,Y_s:\\
    &X_1^{m_1}=X_2^{m_2}=\cdots=X_r^{m_r}=\text{Id},\\
    &X_1X_2\cdots X_rY_1Y_2\cdots Y_sA_1B_1A_1^{-1}B_1^{-1}\cdots A_gB_gA_g^{-1}B_g^{-1}=\text{Id}\rangle
    \end{align*}
    because $X_k$ fixes the point $w_k$ of order $m_k$ and the stabilizer of $v_1$ is trivial.

\end{frame}

\begin{frame}{Poincare Theorem - Example}

	The minimum area possible for a non-cocompact Fuchsian group is attained by the modular group $(0;2,3;1)=(0;2,3,\infty)$
    $$\mu(\H/(0;2,3;1))=\frac{\pi}{3}$$

    The minimum area possible for a cocompact Fuchsian group is attained by $(0;2,3,7)$
    $$\mu(\H/(0;2,3,7))=\frac{\pi}{21}$$

\end{frame}

\begin{frame}{Triangle Groups}

	A Fuchsian group with the signature of the form $(0;m_1,\cdots,m_r;s)$ where $r+s=3$ is called a triangle group. We can also denote its signature with $(0;m_1,m_2,m_3)$ allowing $m_i$ to be infinity. By Theorem \ref{thm:poincare}, a triangle group exists if and only if $$\frac{1}{m_1}+\frac{1}{m_2}+\frac{1}{m_3}<1$$

    There are some interesting examples of triangle groups.

\end{frame}

\begin{frame}{Triangle Groups}

	The figure below shows how a triangle group can be constructed using reflections.
    $$\Gamma=\langle R_1,R_2,R_3: R_1^2=R_2^2=R_3^2=\text{Id}\rangle\cap \PSL_2(\mathbb{R})$$

    \begin{center}
        % \includegraphics[scale=0.3]{figures/reflection-triangle.png}
    \end{center}
    
\end{frame}

\begin{frame}{Triangle Groups}

	\begin{figure}
        \centering
        \begin{tabular}{ccc}
        
        % \includegraphics[width=0.3\textwidth]{figures/hy/2-3-7-blue-v2.png} &
        % \includegraphics[width=0.3\textwidth]{figures/hy/2-3-8-blue.png} & 
        % \includegraphics[width=0.3\textwidth]{figures/hy/2-3-i-black.png} \\
        $(2,3,7)$ & $(2,3,8)$ & $(2,3,\infty)$ \\[6pt]
        
        \end{tabular}
        \caption{Some examples of triangle groups}
        \label{fig:tri}
    \end{figure}

\end{frame}

\begin{frame}{Triangle Groups}

	\begin{figure}
        \centering
        \begin{tabular}{ccc}
        
        % \includegraphics[width=0.3\textwidth]{figures/hy/2-4-5-black.png} &
        % \includegraphics[width=0.3\textwidth]{figures/hy/2-4-6-red.png} & 
        % \includegraphics[width=0.3\textwidth]{figures/hy/2-4-8-black.png} \\
        $(2,4,5)$ & $(2,4,6)$ & $(2,4,8)$ \\[6pt]
        
        \end{tabular}
        \caption{Some examples of triangle groups}
        \label{fig:tri}
    \end{figure}

\end{frame}

\begin{frame}{Triangle Groups}

	\begin{figure}
        \centering
        \begin{tabular}{ccc}
        
        % \includegraphics[width=0.3\textwidth]{figures/hy/2-5-5-black.png} &
        % \includegraphics[width=0.3\textwidth]{figures/hy/2-5-6-black.png} & 
        % \includegraphics[width=0.3\textwidth]{figures/hy/2-5-i-black.png} \\
        $(2,5,5)$ & $(2,5,6)$ & $(2,5,\infty)$ \\[6pt]
        
        \end{tabular}
        \caption{Some examples of triangle groups}
        \label{fig:tri}
    \end{figure}

\end{frame}

\begin{frame}{Triangle Groups}

	\begin{figure}
        \centering
        \begin{tabular}{ccc}
        
        % \includegraphics[width=0.3\textwidth]{figures/hy/2-6-6-red.png} &
        % \includegraphics[width=0.3\textwidth]{figures/hy/2-6-8-black.png} & 
        % \includegraphics[width=0.3\textwidth]{figures/hy/2-i-i-black.png} \\
        $(2,6,6)$ & $(2,6,8)$ & $(2,\infty,\infty)$ \\[6pt]
        
        \end{tabular}
        \caption{Some examples of triangle groups}
        \label{fig:tri}
    \end{figure}

\end{frame}

\begin{frame}{Triangle Groups}

	\begin{figure}
        \centering
        \begin{tabular}{ccc}
        
        % \includegraphics[width=0.3\textwidth]{figures/hy/3-3-4-blue.png} &
        % \includegraphics[width=0.3\textwidth]{figures/hy/5-i-i-black.png} & 
        % \includegraphics[width=0.3\textwidth]{figures/hy/i-i-i-black.png} \\
        $(3,3,4)$ & $(5,\infty,\infty)$ & $(\infty,\infty,\infty)$ \\[6pt]
        
        \end{tabular}
        \caption{Some examples of triangle groups}
        \label{fig:tri}
    \end{figure}

\end{frame}

\appendix

\begin{frame}[allowframebreaks]{References}
	\begin{thebibliography}{9}
        \bibitem{ref:katok} \textbf{Fuchsian Groups}, by Svetlana Katok.
    \end{thebibliography}
\end{frame}

\end{document}
